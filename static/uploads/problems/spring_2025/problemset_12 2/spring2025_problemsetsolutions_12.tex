\documentclass[11pt, a4paper, oneside]{article}

% ===== Page Layout =====
\usepackage[letterpaper,top=2cm,bottom=2cm,left=3cm,right=3cm,marginparwidth=1.75cm]{geometry}
\usepackage{microtype}  % Improved text justification

% ===== Fonts & Encoding =====
\usepackage[T1]{fontenc}
\usepackage[utf8]{inputenc}
\usepackage[english]{babel}
\usepackage{lmodern}

% ===== Math Packages =====
\usepackage{amsmath, amssymb, amsthm}
\usepackage{stmaryrd}
\usepackage{mathrsfs}
\usepackage{bbm}
\usepackage{tensor}
\usepackage{mathtools}

% ===== Graphics & Diagrams =====
\usepackage{graphicx}
\usepackage{tikz}
\usepackage{tikz-cd}
\usepackage{pgfplots}
\pgfplotsset{compat=1.18}
\usepackage{pst-node}

% ===== Bibliography =====
\usepackage{biblatex}
%\addbibresource{references.bib}  % Uncomment and add your .bib file

% ===== Tables =====
\usepackage{makecell}

% ===== Colors =====
\usepackage{xcolor}
\definecolor{linkcolour}{rgb}{0.5,0,0}  % Dark red color for links

% ===== Hyperlinks =====
\usepackage{hyperref}
\hypersetup{
    colorlinks,
    breaklinks,
    urlcolor=linkcolour, 
    linkcolor=linkcolour,
    citecolor=linkcolour
}

% ===== Custom Commands =====
\newcommand{\problem}[1][]{\section{#1} \hfill \par}
\newcommand{\solution}[1][]{\subsection*{#1}\hfill \par}

% ===== Theorem Environments =====
\newtheorem{theorem}{Theorem}
\theoremstyle{remark}
\newtheorem*{remark}{Remark}
\theoremstyle{lemma}
\newtheorem*{lemma}{Lemma}

% ===== Text Highlighting =====
\usepackage{soul}
\newcommand\ba[1]{\setbox0=\hbox{$#1$}%
\rlap{\raisebox{.45\ht0}{\textcolor{linkcolour}{\rule{\wd0}{1pt}}}}#1} 
\def\bc#1{\textcolor{linkcolour}{BC note: {#1}}}
\def\b#1{\textcolor{linkcolour}{{#1}}}

% ===== Comment Environment =====
\usepackage{comment}
\begin{comment}
Useful LaTeX fonts:
\usepackage{mathptmx}
\usepackage{txfonts}
\usepackage{pxfonts}
\usepackage{mathpazo}
\usepackage{mathpple}
\usepackage{kmath,kerkis}
\usepackage{kurier}
\usepackage{arev}
\usepackage{euler}
\usepackage{eulervm}
\end{comment}

\title{Problem Set Week 12 Solutions}
\author{ETHZ Math Olympiad Club}
\date{26 May 2025}
\begin{document}
\maketitle
\problem[Problem ]
Prove that the decimal part of
\[
\frac{5 + \sqrt{26}}{n}
\]
begins with either $n$ zeros or $n$ nines for all positive integers $n$.

\solution[Solution:]

\newpage
\problem[Problem 1 (Bernoulli Competition 2024)]
Let \( A, B \in \mathrm{Mat}_{n \times n}(\mathbb{C}) \). Suppose that every matrix \( C \in \mathrm{Mat}_{n \times n}(\mathbb{C}) \) can be written in the form \( C = AD - DB \) for some \( D \in \mathrm{Mat}_{n \times n}(\mathbb{C}) \). Prove that there exists a polynomial \( R \in \mathbb{C}[x] \) such that
\[
R(A) = A^3 - A^2 - A + I_n \quad \text{and} \quad R(B) = (2 - i)B^4 + (2 + i)B^3 - B.
\]

\solution[Solution:]

\textbf{Step 1.} We prove \( A \) and \( B \) have no common eigenvalues.

By assumption, the linear map
\[
\mathrm{Mat}_{n \times n}(\mathbb{C}) \to \mathrm{Mat}_{n \times n}(\mathbb{C}), \quad X \mapsto AX - XB
\]
is surjective, therefore injective. Suppose \( \lambda \) is a common eigenvalue of \( A \) and \( B \). Then
\[
Av = \lambda v, \qquad w^t B = \lambda w^t
\]
for some \( v, w \neq 0 \). Consider \( X := v w^t \). Then \( X \neq 0 \) and
\[
AX = Av w^t = \lambda v w^t = v \lambda w^t = v w^t B = XB,
\]
which is a contradiction.

\textbf{Step 2.} In fact, we prove that for any two polynomials \( R_1, R_2 \in \mathbb{C}[x] \), there exists a polynomial \( R \in \mathbb{C}[x] \) such that \( R(A) = R_1(A) \) and \( R(B) = R_2(B) \).

Let \( P_i \) be the characteristic polynomial of \( A_i \). Then \( P_i \) and \( P_j \) are relatively prime for \( i \neq j \) (they have no common roots by Step 1). By the Cayley–Hamilton theorem, \( P_i(A_i) = 0 \) for each \( i \).

The Chinese Remainder Theorem in the Euclidean domain \( \mathbb{C}[x] \) implies that there exists a polynomial \( R \in \mathbb{C}[x] \) such that
\[
R \equiv R_i \pmod{P_i} \quad \text{for each } i.
\]
Such an \( R \) satisfies the required property.

\newpage
\problem[Problem ()]
100 prisoners are imprisoned in solitary confinement. Each cell is soundproof and windowless. There is a central living room with a single light bulb, initially turned off. The prisoners cannot see the bulb from their cells.

Each day, the warden selects one prisoner uniformly at random to visit the central room. While in the room, the prisoner may toggle the bulb (on $\leftrightarrow$ off) if they so choose. The prisoner may also choose to declare that \textit{all 100 prisoners have visited the central room at least once}. If this claim is false (i.e., at least one prisoner has never been in the room), all 100 prisoners are executed. If the claim is true, they are all freed and inducted into MENSA.

Before this process begins, the prisoners are allowed to meet once in a courtyard to devise a strategy. What strategy should they adopt to ensure their eventual release, with certainty?

The strategy must guarantee eventual success, not merely a high probability.

The average time until release depends on the strategy.

The best-known strategies achieve an average release time of approximately 3500 days.


\solution[Solution:]

The following is a classical (though not necessarily optimal) solution that guarantees eventual success.
\\\\
The prisoners agree beforehand on a unique individual, called the \textit{counter}. The remaining 99 prisoners are referred to as \textit{non-counters}. The strategy unfolds as follows:
\\\\
Each non-counter prisoner follows this rule:
\begin{itemize}
    \item The \textbf{first time} a non-counter enters the central room and finds the light bulb \textit{off}, they turn it \textit{on}.
    \item They do this \textbf{only once} in the entire process. If the light is already on, or if they've already toggled it once before, they do nothing.
\end{itemize}
The counter follows this rule:
\begin{itemize}
    \item Every time the counter enters the room and finds the bulb \textit{on}, they turn it \textit{off} and increment a mental count by 1.
    \item Once their count reaches 99 (i.e., they have turned the light off 99 times), they confidently declare that all 100 prisoners have visited the room.
\end{itemize}
Each of the 99 non-counters turns the bulb on exactly once, and only upon their \textit{first} encounter with the bulb in the off state. Thus, the counter can infer that each increment in their count corresponds to a distinct non-counter having visited the room. Once the count reaches 99, all non-counters have visited.
\\\\
Since the process is infinite and each prisoner is selected with uniform probability, every prisoner will eventually be selected infinitely often with probability 1. Therefore, the strategy will eventually succeed with certainty.
\\\\
The average number of days until the counter has counted 99 distinct toggles is approximately:
\[
100 \cdot H_{99} \approx 100 \cdot \ln(99) + \gamma \approx 5180 \text{ days},
\]
where \( H_n \) is the \( n \)-th harmonic number and \( \gamma \) is the Euler-Mascheroni constant.
\\\\
Smarter strategies exist (e.g., multiple counters, probabilistic switching, phase-based methods) that reduce the expected time to around 3500 days, but the strategy above is easier to verify and implement.
\\\\
The key idea lies in symmetry-breaking: establishing a unique counter, and having each prisoner signal their participation exactly once. The problem beautifully illustrates distributed consensus under severe communication constraints, and has applications in theoretical computer science and protocol design.


\newpage
\problem[Problem (Hongler)]
Une princesse a $N=100$ prétendants. Chaque prétendant a un "score" différent. Parmi eux, il y a le prince charmant, qui a le score maximal. La princesse rencontre dans un ordre aléatoire les prétendants. Pour chaque prétendant, elle doit décider soit de l'épouser, soit de le rejeter (et si elle le rejette, elle ne peut plus revenir en arrière). Elle peut comparer le score du prétendant qu'elle voit aux scores des prétendants qu'elle a déjà rejetés, mais ne sait pas autrement reconnaître le prince charmant. Elle veut maximiser sa chance d'épouser le prince charmant (si elle arrive au bout de la liste des $N$ prétendants, elle doit se marier avec le dernier).\\
(a) Comment trouver une stratégie qui assure une chance $\geq \frac{1}{4}$ de se marier avec le prince charmant?\\
i. Réponse: elle décide de rejeter les $N / 2$ premiers prétendants et de se marier avec le premier parmi les $N / 2$ derniers qui est mieux que les $N / 2$ premiers prétendants (s'il existe). Cette stratégie a probabilité $\geq \frac{1}{4}$ de fonctionner, car la chance que le dauphin (le préntendant qui est juste en dessous du prince charmant) soit dans le $N / 2$ premiers est $\frac{1}{2}$ et la chance que le prince charmant soit dans les $N / 2$ derniers est $\frac{1}{2}$ aussi. Comme les deux événements sont (presque) indépendants ( $N$ assez grand) la probabilité qu'ils se produisent les deux est $\approx \frac{1}{4}$ et dans ce cas de figure la princesse trouve en effet le prince charmant.\\
(b) Quelle est la stratégie optimale ? (Indice: elle donne une chance $>36 \%$ ) (Indice: la stratégie n'est pas compliquée du tout).\\
i. Stratégie: la princesse laisse tomber les $x$ premiers prétendants et se marie avec le premier qui est mieux que ces $x$ prétendants (s'il existe). En fonction de $x$, quelle est la probabilité $P(x)$ de trouver le prince charmant?\\
ii. Observation: si on regarde la suite des rangs relatifs des prétendants au moment où la princesse les rencontre, elle forme une suite dans

$$
\{1\} \times\{1,2\} \times\{1,2,3\} \times \cdots \times\{1,2, \ldots, N\}
$$

Par symétrie du problème, chaque suite de ce type est équiprobable et a une chance $\frac{1}{N!}$ de se produire.\\
iii. Dans le cas où la princesse se marie avec le prince charmant, les rangs relatifs à partir de $x$ doivent ressembler à cela: il y a un moment $p>x$ où elle voit le prince charmant (qui a donc rang relatif 1) et ensuite, si elle continuait l'expérience, elle ne verrait que des rangs relatifs $>1$.\\
iv. Quelle est la probabilité que cela se produise pour le prétendant $p$ ?

$$
\frac{x}{x+1} \frac{x+1}{x+2} \cdots \frac{p-2}{p-1} \frac{1}{p} \frac{p}{p+1} \frac{p+1}{p+2} \cdots \frac{n-1}{n}=\frac{x}{n(p-1)}
$$

v. Maintenant, il faut sommer sur tous les $p>n$ possibles. Donc on calcule

$$
\begin{aligned}
\sum_{p=x+1}^{n} \frac{x}{n(p-1)} & =\frac{x}{n} \sum_{p=x+1}^{n} \frac{1}{p-1} \\
& =\frac{x}{n} \sum_{p=x}^{n-1} \frac{1}{p} \\
& \approx \frac{x}{n}(\log (n-1)-\log (x)) \\
& =\frac{x}{n} \log \frac{n-1}{x} \approx \frac{x}{n} \log \frac{n}{x} \\
& =-\alpha \log \alpha
\end{aligned}
$$

où $x / n=\alpha$ est la fraction des prétendants que la princesse élimine d'emblée.\\
vi. Comment optimiser par rapport à $\alpha \in[0,1]$ ? Cette fonction fait 0 en 0 et en 1, est positive entre deux... Pour trouver son max, on dérive, on trouve $-\log \alpha-1$. Pour que la dérivée fasse 0 , il faut $\alpha=\frac{1}{e} \approx 37 \%$. Et la valeur de la probabilté en $\alpha=\frac{1}{e}$, est $-\frac{1}{e} \log \left(\frac{1}{e}\right)=\frac{1}{e} \approx 37 \%$.

\newpage
\problem[problem]
(The example comes from the analysis I lecture given by Clement Hongler 2021)
\\\\
Let $\alpha\in\mathbb{R}$ such that $\forall n\in\mathbb{N}^{*}\, n^\alpha:=\exp(\alpha\cdot\ln(n))\in\mathbb{N}^{*}$. Show $\alpha\in\mathbb{N}$.
\solution[Solution:]
Let us to the trivial cases.
\\
If $\alpha<0$ we notice that $\forall n\in\mathbb{N}^{*}$ we have:
$\mathbb{N}\ni n^{\alpha}=\frac{1}{n^{|\alpha|}}$. Since $x\mapsto x^{|\alpha|}:=exp(|\alpha|\cdot ln(x))$ is strictly increasing over $]0;+\infty[$ (it is derivable over $]0;+\infty[$ with positive derivative $|\alpha|\cdot x^{|\alpha|}$) we must have (since $x\mapsto\frac{1}{x}$ is strictly decreasing over $]0;+\infty[$) that $x\mapsto x^{\alpha}$ is strictly decreasing over $]0;+\infty[$. So:
$$1<2\rightarrow 1=1^{\alpha}>2^{\alpha}\in\mathbb{N}\Rightarrow 0=2^{\alpha}=exp(\alpha\cdot ln(2))$$
a contradiction since $0\notin ran(exp)=\mathbb{R}_{>0}$.
\\
Therefore we cannot have $\alpha<0$ and we must have $\alpha\geq 0$.
\\\\
If $\alpha=0$ (this is a valid solution) we are done.
\\\\
If $\alpha>0$, let us get some intuition a bit by doing an informal discussion on how this sentence:
$$\forall n\in\mathbb{N}^{*}\, n^{\alpha}\in\mathbb{N}$$
is intricate with the strictly growing function $x\mapsto x^{\alpha}$ and how the discrete property of the integers $\mathbb{Z}$ and their realisation in the continuum realm of the real number $\mathbb{R}$ can be cleverly used.
\\
For now a simple heuristic shows that if $0<\alpha<1$ then taking the first $2$ number $1,2$ we must have $1=1^{\alpha}$ and $2^\alpha\in\mathbb{N}$ but $ 2\alpha=\exp(\alpha\ln(2))<\exp(1\ln(2))=2^1=2$ so that $2\alpha=1$ but then $\alpha=0$ a contradiction. Thus there is no $\alpha\in]0,1[$ such that the statement holds. Cotninuing on this idea if $1<\alpha<2$ then $1^{\alpha}=1$ and 
\\\\
What is the "minimal" distance one has to do between integer i.e. $(n+1)-n$? Of course it is one unit of distance, lets call this distance the $0$-th order distance. Now, how much this unit distance is being increased when it comes to taking it to the exponent $\alpha$ i.e. $(n+1)^{\alpha}-n^{\alpha}$ (this is a positive value in the $1$-th order case since $x\mapsto x^{\alpha}$ is increasing) ? Knowing a close form of this $1$-th order distance can be helpful because we know $(n+1)^{\alpha},n^{\alpha}\in\mathbb{N}$ and thus this distance being $positive$ should be a strictly positive natural number i.e.  $(n+1)^{\alpha}-n^{\alpha}\in\mathbb{N}^{*}$ ! Imagine if we could quantify that it was $<2$ uniformly over all $n\in\mathbb{N}^{*}$, then we would know $(n+1)^{\alpha}-n^{\alpha}=1$ which in turn would imply (by the binomial theorem) that $\alpha=1$. This idea of trying to force the value of the distance uniformly over the strictly positive natural number is a thing to dig because it is helpful since we are free to roll $n\in\mathbb{N}$ in particular make it tend to $+\infty$.
\\
But what if this distance was $\geq 2$ ? Then we could hope to quantify the $2$-th order distance i.e. the distance between the $1$-th order distance:
$$(((n+1)+1)^{\alpha}-(n+1)^{\alpha})-((n+1)^{\alpha}-n^{\alpha})$$
$$=(((n+2)^{\alpha}-(n+1)^{\alpha})-((n+1)^{\alpha}-n^{\alpha})=(n+2)^{\alpha}+n^{\alpha}-2(n+1)^{\alpha}$$
which this times may be simple an integer in $\mathbb{Z}$, and we can try to get some useful information. We can continue on the $3$-th order distance:
$$\left((((n+1)+2)^{\alpha}-((n+1)+1)^{\alpha})-(((n+1)+1)^{\alpha}-(n+1)^{\alpha})\right)$$
$$-\left(((n+2)^{\alpha}-(n+1)^{\alpha})-((n+1)^{\alpha}-n^{\alpha})\right)$$
$$=(n+3)^{\alpha}-3(n+2)^{\alpha}+(n+1)^{\alpha}-n^{\alpha}\in\mathbb{N}^{*}$$
This starts to go messy...
The key here is to notice that at some point i.e. at some $k$-th order distance and at some rank $n\geq N$ we are \textbf{forced} to take some specific integer value ! And from there we can hope to deduce something about $\alpha$.
\\\\
Surprisingly (or not so surprisingly in fact) we can exactly give a closed form for each of these $k$-th order distance. Remember by using the mean value theorem we have the existence of $\theta_{1}\in]0;1[$ such that:
$$\alpha\cdot (n+\theta_1)^{\alpha-1}=\frac{(n+1)^{\alpha}-n^{\alpha}}{(n+1)-n}=(n+1)^{\alpha}-n^{\alpha}$$
Our above feel that iterating this process ($x\mapsto x^{\alpha}\in C^{\infty}(]0;+\infty[,\mathbb{R})$) until some point is correct. In fact after $\lfloor\alpha\rfloor+1$ iteration of the mean value theorem, the expression on the LHS which really is  $\prod_{i=0}^{\lfloor\alpha\rfloor+1}(\alpha-i)$ multiplied by $(n+\theta_{\lfloor\alpha\rfloor+1})^{(\alpha-(\lfloor\alpha\rfloor+1))<0}$ where $\theta_{\lfloor\alpha\rfloor+1}\in]0;\lfloor\alpha\rfloor+1[$ should represent the $\lfloor\alpha\rfloor+1$-th order distance (which is an integer) ! And this must hold for any $n\in\mathbb{N}$ ! In particular balading $n$ to infinity we must have (as the distance is an integer) that this distance is zero, (this uses $(\alpha-(\lfloor\alpha\rfloor+1))<0$).  This tell us that such a distance at some point must be zero and therefore that $\alpha=\lfloor\alpha\rfloor$ which will show that $\alpha\in\mathbb{N}$ !
\\\\
Let us make this more formally by defining the following operation the \href{https://en.m.wikipedia.org/wiki/Finite_difference}{difference operator }:
\[\begin{tikzcd}
	{\Delta:\mathcal{F}(\mathbb{R},\mathbb{R})} & {\mathcal{F}(\mathbb{R};\mathbb{R})} \\
	f & {\Delta(f):\mathbb{R}} & {\mathbb{R}} \\
	& x & {f(x+1)-f(x)}
	\arrow[from=1-1, to=1-2]
	\arrow[maps to, from=2-1, to=2-2]
	\arrow[from=2-2, to=2-3]
	\arrow[maps to, from=3-2, to=3-3]
\end{tikzcd}\]
It has the property that $\Delta[C^{\infty}(\mathbb{R},\mathbb{R})]\subset C^{\infty}(\mathbb{R},\mathbb{R})$ and that (restricted to $C^{\infty}(\mathbb{R},\mathbb{R})$) the following diagram commutes:
\[\begin{tikzcd}
	{C^{\infty}(\mathbb{R},\mathbb{R})} & {C^{\infty}(\mathbb{R},\mathbb{R})} \\
	{C^{\infty}(\mathbb{R},\mathbb{R})} & {C^{\infty}(\mathbb{R},\mathbb{R})}
	\arrow["\Delta", from=1-1, to=1-2]
	\arrow["{\frac{d}{dx}}"', from=1-1, to=2-1]
	\arrow["{\frac{d}{dx}}", from=1-2, to=2-2]
	\arrow["\Delta"', from=2-1, to=2-2]
\end{tikzcd}\]
Indeed for $f\in C^{\infty}(\mathbb{R},\mathbb{R})$ and $\tilde{x}\in\mathbb{R}$ we have:
$$\left((\frac{d}{dx}\circ\Delta) (f)\right)(\tilde{x})=\left(\frac{d}{dx}(\Delta(f))\right)(\tilde{x}+1)$$
$$=\left(\frac{d}{dx}(f)\right)(\tilde{x}+1)\cdot \left(\frac{d}{dx}(\_+1)\right)(\tilde{x})-\left(\frac{d}{dx}(f)\right)(\tilde{x})=\left(\frac{d}{dx}(f)\right)(\tilde{x}+1)-\left(\frac{d}{dx}(f)\right)(\tilde{x})$$
$$=\left(\Delta\left(\frac{d}{dx}(f)\right)\right)(\tilde{x})=\left((\Delta\circ\frac{d}{dx})(f)\right)(\tilde{x})$$
We define by recursion along the integers; for $k\in\mathbb{N}$:
$$\Delta^{(k)}=\begin{cases}
    Id\text{ if } k = 0\\
    \Delta\circ \Delta^{(k-1)}\text{ if } k\geq 1
\end{cases}$$
and we define the set:
$$\mathcal{A}:=\left\{k\in\mathbb{N}\,:\, \forall f\in C^{\infty}(\mathbb{R},\mathbb{R})\forall x\in\mathbb{R}\exists\theta_{k}\in]0;k[\, \left(\frac{d^k}{dx^k}(f)\right)(x+\theta_{k})=(\Delta^{(k)}(f))(x)\right\}$$
(This is just formal induction)
\\We claim $\mathcal{A}=\mathbb{N}$. One inclusion is trivial. For the other inclusion: suppose $\mathbb{N}\setminus\mathcal{A}\neq\varnothing$ then we can take:
$$m:=min\, \mathbb{N}\setminus\mathcal{A}$$
As for any function $f\in C^{\infty}(\mathbb{R},\mathbb{R})$ we have:
$$\left(\frac{d^0}{dx^0}(f)\right):=f=Id(f)=(\Delta^{(0)}(f))$$ this means that $0\in\mathcal{A}$ so $m\neq 0$ i.e. $m$ is a succesor natural number with $m=(m-1)+1$ and $m-1\in\mathcal{A}$ (by minimality). Fix $f\in C^{\infty}(\mathbb{R},\mathbb{R})$ by hypothesis we know 
that for $\frac{d}{dx}(f)\in C^{\infty}(\mathbb{R},\mathbb{R})$ that $\exists \theta_{m-1}\in\mathcal{F}(\mathbb{R};]0;m-1[)$ with $\forall \tilde{x}\in\mathbb{R}$:
$$\left(\frac{d^{m-1}}{dx^{m-1}}\left(\frac{d}{dx}(f)\right)\right)(\tilde{x}+\theta_{m-1}(\tilde{x}))=\left(\Delta^{(m-1)}\left(\frac{d}{dx}(f)\right)\right)(\tilde{x})
$$
Also as $\Delta[C^{\infty}(\mathbb{R},\mathbb{R})]\subset C^{\infty}(\mathbb{R},\mathbb{R})$ we must have $\Delta^{(m-1)}(f)\in C^{\infty}(\mathbb{R};\mathbb{R})$. Fixing $\tilde{x}\in\mathbb{R}$, so we can apply the mean value theorem to obtain $\exists u\in ]0;1[$ such that:
$$\left(\Delta^{(m)}(f)\right)(\tilde{x})\overset{def}{=}\left(\Delta^{(m-1)}(f)\right)(\tilde{x}+1)-\left(\Delta^{(m-1)}(f)\right)(\tilde{x})$$
$$\frac{\left(\Delta^{(m-1)}(f)\right)(\tilde{x}+1)-\left(\Delta^{(m-1)}(f)\right)(\tilde{x})}{(\tilde{x}+1)-\tilde{x}}\overset{mean}{=}\left(\frac{d}{dx}\left(\Delta^{(m-1)}(f)\right)\right)(\tilde{x}+u)$$
$$\overset{\frac{d}{dx}\circ\Delta=\Delta\circ\frac{d}{dx}}{=}\left(\Delta^{(m-1)}\left(\frac{d}{dx}(f)\right)\right)(\tilde{x}+u)\overset{hyp}{=}\left(\frac{d^{m-1}}{dx^{m-1}}\left(\frac{d}{dx}(f)\right)\right)(\tilde{x}+u+\theta_{m-1}(\tilde{x}+u))$$
$$=\left(\frac{d^{m}}{dx^{m}}(f)\right)(\tilde{x}+u+\theta_{m-1}(\tilde{x}+u))$$
Since $u+\theta_{m-1}(\tilde{x}+u)\in]0;m[$ we conclude. As $\tilde{x}\in\mathbb{R}$ and $f\in C^{\infty}(\mathbb{R},\mathbb{R})$ were arbitrary we conclude $m\in\mathcal{A}$. This is a contradiction and $\mathcal{A}=\mathbb{N}$ as desired.
\\\\
Take the function:
$$f(x):=\begin{cases}
x^{\alpha} \text{ if } x\in\mathbb{R}_{+}    \\
0 \text{ else }
\end{cases}
$$
we have $f\in C^{\infty}(\mathbb{R},\mathbb{R})$ (not analytic!), so we can apply the theorem to obtain that for $\lfloor\alpha\rfloor+1\in\mathbb{N}=\mathcal{A}$ and  $\forall n\in\mathbb{N}^{*}\exists\theta_{\lfloor\alpha\rfloor+1}\in]0;\lfloor\alpha\rfloor+1[$ with:
$$\left(\Delta^{(\lfloor\alpha\rfloor+1)}(f)\right)(n)=\left(\frac{d^{\lfloor\alpha\rfloor+1}}{dx^{\lfloor\alpha\rfloor+1}}f\right)(n+\theta_{\lfloor\alpha\rfloor+1})$$
$$=\left(\prod_{i=0}^{\lfloor\alpha\rfloor}(\alpha-i)\right)\frac{1}{(n+\theta_{\lfloor\alpha\rfloor+1})^{\lfloor\alpha\rfloor+1-\alpha}}\in\left[0;\left(\prod_{i=0}^{\lfloor\alpha\rfloor}(\alpha-i)\right)\left(\frac{1}{n}\right)^{\lfloor\alpha\rfloor+1-\alpha}\right]$$
Since $$lim_{n\rightarrow+\infty}\left(\prod_{i=0}^{\lfloor\alpha\rfloor}(\alpha-i)\right)\left(\frac{1}{n}\right)^{\lfloor\alpha\rfloor+1-\alpha}\overset{\lfloor\alpha\rfloor+1-\alpha>0}{=}0$$
We must have by the squeeze theorem:
$$lim_{n\rightarrow+\infty}\left(\Delta^{(\lfloor\alpha\rfloor+1)}(f)\right)(n)=0$$
However by induction we can show that $\forall k\in\mathbb{N}$ we must have $(\Delta^{(k)}(f))[\mathbb{Z}]\subset\mathbb{Z}$. Combining these two fact we have that for some $N\in\mathbb{N}$ the following phenomena occurs:
$$0\leq (\Delta^{(\lfloor\alpha\rfloor+1)}(f))(N)=\left(\prod_{i=0}^{\lfloor\alpha\rfloor}(\alpha-i)\right)\left(\frac{1}{N}\right)^{\lfloor\alpha\rfloor+1-\alpha}<\frac{1}{2}$$
and:$$(\Delta^{(\lfloor\alpha\rfloor+1)}(f))(N)\in\mathbb{Z}$$
Hence $\left(\prod_{i=0}^{\lfloor\alpha\rfloor}(\alpha-i)\right)\left(\frac{1}{N}\right)^{\lfloor\alpha\rfloor+1-\alpha}=0$ so $\left(\prod_{i=0}^{\lfloor\alpha\rfloor}(\alpha-i)\right)=0$ and as $i<\lfloor\alpha\rfloor\rightarrow \alpha\neq i$ this implies $\alpha=\lfloor\alpha\rfloor$.
So in total we have shown (at least in \textbf{ZFC}):
$$\forall\alpha\in\mathbb{R}\left((\forall n\in\mathbb{N}^{*}\, n^{\alpha}\in\mathbb{N})\rightarrow \alpha\in\mathbb{N}\right)$$
Remark: One has use the fact we could take $n$ to infinity, one can ask if we can reduce the set $\mathbb{N}^{*}$ say to an unbounded part of $\mathbb{N}^{*}$ or even to a finite set (but then this would require other tricks). A difficult result $6$ exponent theorem proved in 1966 tells us that the following formula holds:
$$\forall\alpha\in\mathbb{R}\left((\forall n\in\{2,3,5\}\, n^{\alpha}\in\mathbb{N})\rightarrow \alpha\in\mathbb{N}\right)$$
We have not decided yet the $4$ exponent conjecture/ Schanuel conjecture:
$$\forall\alpha\in\mathbb{R}\left((\forall n\in\{2,3\}\, n^{\alpha}\in\mathbb{N})\rightarrow \alpha\in\mathbb{N}\right)$$
(a set of size $1$ is false for instance $4^{\frac{1}{2}}=2$). \\


\end{document}
