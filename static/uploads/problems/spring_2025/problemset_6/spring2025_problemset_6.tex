\documentclass[11pt, a4paper, oneside]{article}

% ===== Page Layout =====
\usepackage[letterpaper,top=2cm,bottom=2cm,left=3cm,right=3cm,marginparwidth=1.75cm]{geometry}
\usepackage{microtype}  % Improved text justification

% ===== Fonts & Encoding =====
\usepackage[T1]{fontenc}
\usepackage[utf8]{inputenc}
\usepackage[english]{babel}
\usepackage{lmodern}

% ===== Math Packages =====
\usepackage{amsmath, amssymb, amsthm}
\usepackage{stmaryrd}
\usepackage{mathrsfs}
\usepackage{bbm}
\usepackage{tensor}
\usepackage{mathtools}

% ===== Graphics & Diagrams =====
\usepackage{graphicx}
\usepackage{tikz}
\usepackage{tikz-cd}
\usepackage{pgfplots}
\pgfplotsset{compat=1.18}
\usepackage{pst-node}

% ===== Bibliography =====
\usepackage{biblatex}
\addbibresource{references.bib}  % Uncomment and add your .bib file

% ===== Tables =====
\usepackage{makecell}

% ===== Colors =====
\usepackage{xcolor}
\definecolor{linkcolour}{rgb}{0.5,0,0}  % Dark black color for links

% ===== Hyperlinks =====
\usepackage{hyperref}
\hypersetup{
    colorlinks,
    breaklinks,
    urlcolor=linkcolour, 
    linkcolor=linkcolour,
    citecolor=linkcolour
}

% ===== Custom Commands =====
\newcommand{\problem}[1][]{\section{#1} \hfill \par}
\newcommand{\answer}{\subsection*{Answer:}\hfill \par}

% ===== Theorem Environments =====
\newtheorem{theorem}{Theorem}
\theoremstyle{remark}
\newtheorem*{remark}{Remark}

% ===== Text Highlighting =====
\usepackage{soul}
\newcommand\ba[1]{\setbox0=\hbox{$#1$}%
\rlap{\raisebox{.45\ht0}{\textcolor{linkcolour}{\rule{\wd0}{1pt}}}}#1} 
\def\bc#1{\textcolor{linkcolour}{BC note: {#1}}}
\def\b#1{\textcolor{linkcolour}{{#1}}}

% ===== Comment Environment =====
\usepackage{comment}
\begin{comment}
Useful LaTeX fonts:
\usepackage{mathptmx}
\usepackage{txfonts}
\usepackage{pxfonts}
\usepackage{mathpazo}
\usepackage{mathpple}
\usepackage{kmath,kerkis}
\usepackage{kurier}
\usepackage{arev}
\usepackage{euler}
\usepackage{eulervm}
\end{comment}

\title{Problem Set Week 6}
\author{ETHZ Math Olympiad Club}
\date{31 March 2025}
\begin{document}
\maketitle
\problem[Problem B-1 (IMC 2023)]
Ivan writes the matrix
\[
A = \begin{bmatrix} 2 & 2 \\ 3 & 4 \end{bmatrix}
\]
on the board. Then he performs the following operation on the matrix several times:
\begin{itemize}
    \item He chooses a row or a column of the matrix, and
    \item He multiplies or divides the chosen row or column entry-wise by the other row or column, respectively.
\end{itemize}
Can Ivan end up with the matrix
\[
B = \begin{bmatrix} 2 & 2 \\ 4 & 3 \end{bmatrix}
\]
after finitely many steps?
\problem[Vieta Jumping Problems]
\subsection{Problem 6 (IMO 1988)}
Let \( a \) and \( b \) be positive integers such that \( ab + 1 \) divides \( a^{2} + b^{2} \). Show that
\[ \frac{a^{2} + b^{2}}{ab + 1} \]
is the square of an integer.
\subsection{Problem (Kevin Buzzard \& Edward Crane)}
Let $a$ and $b$ be positive integers. Show that if $4ab - 1$ divides $\left(4a^{2} - 1\right)^{2}$, then $a = b$. 
\problem[Problem A-3 (IMC 2015)] Let $F(0) = 0$, $F(1) =\frac{3}{2}$, and 
\[
F(n) = \frac{5}{2}F(n-1) - F(n-2) \quad \text{for } n \geq 2.
\]
Determine whether or not
\[
\sum_{n=0}^{\infty} \frac{1}{F(2^n)}
\]
is a rational number.


\end{document}