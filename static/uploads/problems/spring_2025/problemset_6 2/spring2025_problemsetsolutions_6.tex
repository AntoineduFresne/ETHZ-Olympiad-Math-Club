\documentclass[11pt, a4paper, oneside]{article}

% ===== Page Layout =====
\usepackage[letterpaper,top=2cm,bottom=2cm,left=3cm,right=3cm,marginparwidth=1.75cm]{geometry}
\usepackage{microtype}  % Improved text justification

% ===== Fonts & Encoding =====
\usepackage[T1]{fontenc}
\usepackage[utf8]{inputenc}
\usepackage[english]{babel}
\usepackage{lmodern}

% ===== Math Packages =====
\usepackage{amsmath, amssymb, amsthm}
\usepackage{stmaryrd}
\usepackage{mathrsfs}
\usepackage{bbm}
\usepackage{tensor}
\usepackage{mathtools}

% ===== Graphics & Diagrams =====
\usepackage{graphicx}
\usepackage{tikz}
\usepackage{tikz-cd}
\usepackage{pgfplots}
\pgfplotsset{compat=1.18}
\usepackage{pst-node}

% ===== Bibliography =====
\usepackage{biblatex}
\addbibresource{references.bib}  % Uncomment and add your .bib file

% ===== Tables =====
\usepackage{makecell}

% ===== Colors =====
\usepackage{xcolor}
\definecolor{linkcolour}{rgb}{0.5,0,0}  % Dark black color for links

% ===== Hyperlinks =====
\usepackage{hyperref}
\hypersetup{
    colorlinks,
    breaklinks,
    urlcolor=linkcolour, 
    linkcolor=linkcolour,
    citecolor=linkcolour
}

% ===== Custom Commands =====
\newcommand{\problem}[1][]{\section{#1} \hfill \par}
\newcommand{\solution}[1][]{\subsection*{#1}\hfill \par}

% ===== Theorem Environments =====
\newtheorem{theorem}{Theorem}
\theoremstyle{remark}
\newtheorem*{remark}{Remark}
\theoremstyle{lemma}
\newtheorem*{lemma}{Lemma}

% ===== Text Highlighting =====
\usepackage{soul}
\newcommand\ba[1]{\setbox0=\hbox{$#1$}%
\rlap{\raisebox{.45\ht0}{\textcolor{linkcolour}{\rule{\wd0}{1pt}}}}#1} 
\def\bc#1{\textcolor{linkcolour}{BC note: {#1}}}
\def\b#1{\textcolor{linkcolour}{{#1}}}

% ===== Comment Environment =====
\usepackage{comment}
\begin{comment}
Useful LaTeX fonts:
\usepackage{mathptmx}
\usepackage{txfonts}
\usepackage{pxfonts}
\usepackage{mathpazo}
\usepackage{mathpple}
\usepackage{kmath,kerkis}
\usepackage{kurier}
\usepackage{arev}
\usepackage{euler}
\usepackage{eulervm}
\end{comment}


\title{Problem Set Week 6 Solutions}
\author{ETHZ Math Olympiad Club}
\date{31 March 2025}
\begin{document}
\maketitle
\problem[Problem B-1 (IMC 2023)]
Ivan writes the matrix
\[
A = \begin{bmatrix} 2 & 2 \\ 3 & 4 \end{bmatrix}
\]
on the board. Then he performs the following operation on the matrix several times:
\begin{itemize}
    \item He chooses a row or a column of the matrix, and
    \item He multiplies or divides the chosen row or column entry-wise by the other row or column, respectively.
\end{itemize}
Can Ivan end up with the matrix
\[
B = \begin{bmatrix} 2 & 2 \\ 4 & 3 \end{bmatrix}
\]
after finitely many steps?


\solution[Solution:] 
We show that starting from \( A = \begin{bmatrix} 2 & 2 \\ 3 & 4 \end{bmatrix} \), Ivan cannot reach the matrix \( B = \begin{bmatrix} 2 & 2 \\ 4 & 3 \end{bmatrix} \).
\\\\
Notice first that the allowed operations preserve the positivity of entries; all matrices Ivan can reach have only positive entries. The key insight is to recognize that the operations resemble standard row/column addition/subtraction operations (which preserve determinants), but here addition/subtraction is replaced by multiplication/division. This suggests the need for a group morphism. Hence, we define boldly for any matrix \( X = \begin{bmatrix} x_{11} & x_{12} \\ x_{21} & x_{22} \end{bmatrix} \in \mathbb{R}_{>0}^{2 \times 2} \) with positive entries, the following auxiliary logarithmic transformation matrix:
\[
L(X) = \begin{bmatrix} \log_2\left(x_{11}\right) & \log_2\left(x_{12}\right) \\ \log_2\left(x_{21}\right) & \log_2\left(x_{22}\right) \end{bmatrix}.
\]
By taking logarithms of the entries, which transform multiplication to addition (as it is a group morphism from \( (\mathbb{R}_{>0},\cdot) \) to \( (\mathbb{R},+) \)), Ivan's operations on \( L(X) \) translate to adding or subtracting a row or column of \( L(X) \) to itself. Such standard row and column operations are well known to preserve the determinant. Hence, if Ivan performs one operation on matrix \( X_0 \) to obtain \( X_1 \), then:
\[
{\det}_{\mathbb{R}}\left(L(X_0)\right)= {\det}_{\mathbb{R}}\left(L(X_1)\right).
\]
By a trivial induction, for any finite sequence of operations of length \( n \geq 0 \), applied recursively starting from \( X_0 \), we obtain the sequence of matrices \( X_0, \dots, X_n \) for which we have:
\[
{\det}_{\mathbb{R}}\left(L(X_0)\right)= {\det}_{\mathbb{R}}\left(L(X_n)\right).
\]
Thus, a necessary condition to reach \( B \) from \( A \) with a finite sequence of operations on \( A \) is that the corresponding logarithmic matrix satisfies:
\[
{\det}_{\mathbb{R}}\left(L(A)\right)= {\det}_{\mathbb{R}}\left(L(B)\right).
\]
However,
\[
{\det}_{\mathbb{R}}\left(L(A)\right) = \log_2\left(2\right) \cdot \log_2\left(4\right) - \log_2\left(2\right) \cdot \log_2\left(3\right) = \log_2\left(\frac{4}{3}\right) > \log_2\left(1\right) = 0,
\]
and
\[
{\det}_{\mathbb{R}}\left(L(B)\right) = \log_2\left(2\right) \cdot \log_2\left(3\right) - \log_2\left(2\right) \cdot \log_2\left(4\right) = \log_2\left(\frac{3}{4}\right) < \log_{2}\left(1\right) = 0.
\]
That is \( {\det}_{\mathbb{R}}\left(L(B)\right) < 0 < {\det}_{\mathbb{R}}\left(L(A)\right) \) and the two quantities cannot be equal. This proves that Ivan cannot transform \( A \) into \( B \).
\newpage
\problem[Vieta Jumping Problems]
\subsection{Problem 6 (IMO 1988)}
Let \( a \) and \( b \) be positive integers such that \( ab + 1 \) divides \( a^{2} + b^{2} \). Show that
\[ \frac{a^{2} + b^{2}}{ab + 1} \]
is the square of an integer.

\solution[Solution:]
Define
\[
S := \left\{ (a,b) \in \mathbb{N}^{*} \times \mathbb{N}^{*} \mid ab+1 \mid a^2 + b^2 \right\}.
\]
For each \( k \in \mathbb{N}^{*} \), define
\[
S_k := \left\{ (a,b) \in \mathbb{N}^{*} \times \mathbb{N}^{*} \mid a^2 + b^2 = k(ab+1) \right\}.
\]
Clearly, \( S \) is decomposed into the sets \( S_k \):
\[
S = \bigcup_{k \in \mathbb{N}^{*}} S_k,
\]
and for any \( k \in \mathbb{N}^{*} \) we have:
\[
(a,b) \in S_k \Leftrightarrow (b,a) \in S_k.
\]
To show the statement of the problem, we therefore need to show that for a fixed \( k > 0 \) with \( S_k \neq \varnothing \), we have \( k \in \square_{\mathbb{N}} := \left\{ n^2 \mid n \in \mathbb{N} \right\} \).
\\\\
For a fixed \( k > 0 \) with \( S_k \neq \varnothing \), we can define the following minimum:
\[
b' := \min \left\{ \min\{a,b\} \mid (a,b) \in S_k \right\}.
\]
By construction, there exists \( (a,b) \in S_k \) such that \( b' = \min\{a,b\} \). Label
\[
a' = \max\{a, b\}.
\]
Then \( (a',b') \in S_k \) and we have the equation:
\[
a'^2 - k a' b' + b'^2 - k = 0,
\]
which is quadratic in \( a' \). Let \( c' \in \mathbb{C} \) be the other root of the polynomial \( X^2 - k b' X + b'^2 - k \) (this is the famous \textit{root jumping}). We show that \( c' = 0 \), leading to the factorization
\[
X^2 - k b' X + b'^2 - k = X(X - a'),
\]
and will imply that \( k = b'^2 \), which will conclude $k\in\square_{\mathbb{N}}$.
\\\\
By Vieta's formulas for the coefficient of degree 1 and the constant coefficient, we obtain respectively:
\[
c' = k b' - a' \in \mathbb{Z},
\]
and
\[
a' c' = b'^2 - k < b'^2,
\]
where we used the fact \( k \geq 1 \). In particular, \( c' \) is subject to the order \( < \) of \( \mathbb{Z} \). Suppose to obtain a contradiction that \( c' > 0 \); then:
\[
b' c' \leq a' c' < b'^2 \implies c' < b',
\]
where we used \( 0 < b' \leq a' \) and \( 0 < c' \).
\\\\
By choice of \( c' \), we have \( c'^2 - k b' c' + b'^2 - k = 0 \), i.e.,
\[
c'^2 + b'^2 = k(c' b' + 1),
\]
this implies that \( (c',b') \in S_k \). As \( c' < b' \), we have a contradiction to the minimality of \( b' \). Thus \( c' \leq 0 \). 
\\\\
Additionally,
\[
(a' + 1)(c' + 1) = a' c' + a' + c' + 1 = b'^2 - k + b' k + 1 = b'^2 + (b' - 1)k + 1 \geq 1,
\]
where we used \( b' \geq 1 \) and \( k > 0 \). As \( a' + 1 > 1 \), we must have \( c' + 1 > 0 \), i.e., \( c' > -1 \), so we conclude that \( c' = 0 \) as desiblack. Since \( k > 0 \) with \( S_k \neq \varnothing \) was arbitrary, we are done.


\subsection{Problem (Kevin Buzzard \& Edward Crane)}
Let $a$ and $b$ be positive integers. Show that if $4ab - 1$ divides $\left(4a^{2} - 1\right)^{2}$, then $a = b$. 

\solution[Solution:]

Standard manipulations\footnote{Let $m = 4ab - 1$. Since \(\gcd(b,m)=1\), \(b\) is invertible modulo \(m\).  Note that
\[
4a^2b \;=\;a\,(4ab)\;\equiv\;a\pmod{m}
\quad\Longrightarrow\quad
4a^2 \equiv a\,b^{-1} \pmod{m}.
\]
Hence $b\,(4a^2 - 1)\equiv b\,(ab^{-1}-1)\equiv a - b \pmod{m}
\quad\Longrightarrow\quad
b^2\,(4a^2 - 1)^2 \equiv (a-b)^2 \pmod{m}$. By hypothesis \(m\mid (4a^2-1)^2\), so \(m\mid b^2(4a^2-1)^2\).  Therefore, $4ab-1 \mid (a-b)^2$, as claimed.} of $4ab - 1 \mid \left(4a^{2} - 1\right)^{2}$ show that this implies $4ab - 1 \mid (a - b)^{2}$.
\\
\\
So it suffices to show the statement that if $4ab - 1 \mid_{\mathbb{Z}} (a - b)^2$ then $a = b$. Assume for the sake of contradiction that there exist positive integers $a$ and $b$ with $a \neq b$ such that $4ab - 1 \mid (a - b)^{2}$. Define:
\[
k = \frac{(a - b)^{2}}{4ab - 1} > 0.
\]
Consider the set
\[
S_k := \left\{ (a', b') \in \mathbb{N}^{*} \times \mathbb{N}^{*} \mid (a' - b')^{2} = k(4a'b' - 1) \right\}.
\]
Notice that $S_k$ cannot contain a pair on the diagonal of $\mathbb{N}^{*}$ as $4a'b' - 1 \geq 3$ and $k > 0$ and, in addition, $(a', b') \in S_k \Leftrightarrow (b', a') \in S_k$.
\\\\
Since $S_k \neq \varnothing$, there exists a pair $(A, B) \in S_k$ that minimizes $A + B$.  As said above, we necessarily have $A \neq B$ and by the symmetry above we can without loss of generality assume $A > B$. Consider the quadratic polynomial arising from the condition of $(A, B)\in S_k$:
\[
X^{2} - (2B + 4kB)X + B^{2} + k = 0.
\]
This polynomial has roots $x_1 := A > 0$ by construction. By Vieta's formulas for the coefficient of degree 1 and the constant coefficient, we obtain respectively that:
\[
x_2 := 2B + 4kB - A \in \mathbb{Z}
\]
is also a root (this is the famous \textit{root jumping}) and an integer, satisfying:
\[
x_2 = \frac{B^{2} + k}{A} > 0.
\]
Since $x_2$ is a positive integer, we have $(x_2, B) \in S_k$. By the minimality of $A + B$: $A + B \leq x_2 + B$ from which it follows $x_2 \geq A$, i.e.,
\[
\frac{B^{2} + k}{A} \geq A.
\]
Thus, we obtain (as $A>0$)
\[
k \geq A^{2} - B^{2} = (A - B)(A + B)>0.
\]
From $k\geq (A - B)(A + B)>0$ and $4AB-1\geq 3$ we get in order:
\[
(A - B)^{2} = k(4AB - 1)\geq \left(A^{2} - B^{2}\right)3.
\]
Since $A-B>0$ we can divide the inequality on both sides without changing the inequality direction to obtain:
\[
A + B \geq A - B \geq 3(A + B) \geq 3(A + B),
\]
Again dividing by $A+B>0$ the inequality is preserved and we obtain $1 \geq 3$. This is a contradiction. Thus, our assumption must be false, and the statement is proven.

\newpage
\problem[Problem A-3 (IMC 2015)] Let $F(0) = 0$, $F(1) =\frac{3}{2}$, and 
\[
F(n) = \frac{5}{2}F(n-1) - F(n-2) \quad \text{for } n \geq 2.
\]
Determine whether or not
\[
\sum_{n=0}^{\infty} \frac{1}{F(2^n)}
\]
is a rational number.

\solution[Solution:]
The characteristic polynomial of the linear recurrence is:
\[
X^2 - \frac{5}{2}X + 1,
\]
which has roots $\{2,\frac{1}{2}\}$. Thus, the general form of $F(n)$ is given by:
\[
F(n) = a \cdot 2^n + b \cdot \left(\frac{1}{2}\right)^n
\]
for some constants $a$ and $b$. Using the initial conditions $F(0) = 0$ and $F(1) = \frac{3}{2}$, we obtain the system:
\[
a + b = 0,
\]
\[
2a + \frac{b}{2} = \frac{3}{2}.
\]
Solving for $a$ and $b$, we get $a = 1$, $b = -1$. Therefore,
\[
F(n) = 2^n - 2^{-n}
\]
which is strictly greater than $0$ as long as $n\geq 1$.
Now, we can rewrite the summand:
\[
\frac{1}{F(2^n)} = \frac{1}{2^{2^n} - 2^{-2^n}} = \frac{2^{2^n}}{(2^{2^n})^2 - 1}=\frac{1}{2^{2^n} - 1} - \frac{1}{2^{2^{n+1}} - 1},
\]
where we use in the last equality the fact that for $a\in\mathbb{R}\setminus\{\pm 1\}$ we have: $$\frac{a}{a^2-1}=\frac{a}{a+1}\frac{1}{a-1}=\left(1-\frac{1}{a+1}\right)\frac{1}{a-1}=\frac{1}{a-1}-\frac{1}{a^2-1}.$$
We obtain clearly a telescoping sum:
\[
\sum_{n=0}^{\infty} \frac{1}{F(2^n)} = \sum_{n=0}^{\infty} \left( \frac{1}{2^{2^n} - 1} - \frac{1}{2^{2^{n+1}} - 1} \right) = \frac{1}{2^{2^0} - 1} = 1.
\]
Since $1$ is rational, the sum is rational.

\end{document}