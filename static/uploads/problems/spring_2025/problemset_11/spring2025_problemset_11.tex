\documentclass[11pt, a4paper, oneside]{article}

% ===== Page Layout =====
\usepackage[letterpaper,top=2cm,bottom=2cm,left=3cm,right=3cm,marginparwidth=1.75cm]{geometry}
\usepackage{microtype}  % Improved text justification

% ===== Fonts & Encoding =====
\usepackage[T1]{fontenc}
\usepackage[utf8]{inputenc}
\usepackage[english]{babel}
\usepackage{lmodern}

% ===== Math Packages =====
\usepackage{amsmath, amssymb, amsthm}
\usepackage{stmaryrd}
\usepackage{mathrsfs}
\usepackage{bbm}
\usepackage{tensor}
\usepackage{mathtools}

% ===== Graphics & Diagrams =====
\usepackage{graphicx}
\usepackage{tikz}
\usepackage{tikz-cd}
\usepackage{pgfplots}
\pgfplotsset{compat=1.18}
\usepackage{pst-node}

% ===== Bibliography =====
\usepackage{biblatex}
%\addbibresource{references.bib}  % Uncomment and add your .bib file

% ===== Tables =====
\usepackage{makecell}

% ===== Colors =====
\usepackage{xcolor}
\definecolor{linkcolour}{rgb}{0.5,0,0}  % Dark red color for links

% ===== Hyperlinks =====
\usepackage{hyperref}
\hypersetup{
    colorlinks,
    breaklinks,
    urlcolor=linkcolour, 
    linkcolor=linkcolour,
    citecolor=linkcolour
}

% ===== Custom Commands =====
\newcommand{\problem}[1][]{\section{#1} \hfill \par}
\newcommand{\solution}[1][]{\subsection*{#1}\hfill \par}

% ===== Theorem Environments =====
\newtheorem{theorem}{Theorem}
\theoremstyle{remark}
\newtheorem*{remark}{Remark}
\theoremstyle{lemma}
\newtheorem*{lemma}{Lemma}

% ===== Text Highlighting =====
\usepackage{soul}
\newcommand\ba[1]{\setbox0=\hbox{$#1$}%
\rlap{\raisebox{.45\ht0}{\textcolor{linkcolour}{\rule{\wd0}{1pt}}}}#1} 
\def\bc#1{\textcolor{linkcolour}{BC note: {#1}}}
\def\b#1{\textcolor{linkcolour}{{#1}}}

% ===== Comment Environment =====
\usepackage{comment}
\begin{comment}
Useful LaTeX fonts:
\usepackage{mathptmx}
\usepackage{txfonts}
\usepackage{pxfonts}
\usepackage{mathpazo}
\usepackage{mathpple}
\usepackage{kmath,kerkis}
\usepackage{kurier}
\usepackage{arev}
\usepackage{euler}
\usepackage{eulervm}
\end{comment}

\title{Problem Set Week 11}
\author{ETHZ Math Olympiad Club}
\date{19 May 2025}
\begin{document}
\maketitle
\problem[Problem 1 (Bernoulli Competition 2023)]
1. Let \( A = \{1, 2, \ldots, 100\} \) be the set of integers between 1 and 100.

(a) Let \( B \subset A \) be a subset that doesn't contain two consecutive integers. What is the maximal cardinality of \( B \)?

(b) Let \( C \subset A \) be a subset such that there is no \( n \) for which \( n \) and \( 2n \) are both in \( C \). What is the maximal cardinality of \( C \)?

\problem[Problem (Selected Real Analysis Problem)]
For each function \(g\in\{-id_{\mathbb{R}}, \exp, x\mapsto x^2-2\}\), determine wheter there exists a continuous function $f:\mathbb{R}\rightarrow\mathbb{R}$ such that $f\circ f = g$.
\\\\
\textbf{Bonus:} Solve the same problem with $g\in\{\cos,\sin\}$.

\problem[Problem B4 (Putnam 2001)]
Let \( S:=\mathbb{Q}\setminus\{-1,0-1\} \). Define \( f : S \to S \) by  
\( f(x) = x - \frac{1}{x} \). Prove or disprove that  
\[
\bigcap_{n=1}^{\infty} f^{(n)}(S) = \emptyset,
\]
where \( f^{(n)} \) denotes \( f \) composed with itself \( n \) times.

\problem[Problem (Hongler)]
We have a linked list of elements, \(x_{0}, x_{1}, \ldots, x_{n}\), where \(n\) is unknown (but it is known that the list is finite). When at \(x_{0}\), we have a pointer to go to \(x_{1}\), which leads to \(x_{2}\), and so on, until we reach \(x_{n}\), where we learn that it is the end. We have a bounded amount of memory (we cannot simply store the entire list in an array and choose an element from the array once we reach the end).\\
(a) How can we select a random element from the list, uniformly, if we are allowed to traverse the list only once?\\
(b) Why might solving this problem be useful in practice?\\



\problem[Problem (Hongler)]
Let \( U \subset \mathbb{C} \) be a domain containing the disc \( \mathbb{D} \); that is, \( \mathbb{D} \subset U \), and let \( f : U \to \mathbb{C} \) be a holomorphic function. Show that if \( f\left( \partial \mathbb{D} \right) = \gamma \) is a simple loop (i.e., it can be parametrised by a continuous path which intersects only at the endpoints), and if \( f|_{\partial \mathbb{D}} : \partial \mathbb{D} \to \gamma \) is injective, then \( f|_{\mathbb{D}} \) is injective.






\end{document}
