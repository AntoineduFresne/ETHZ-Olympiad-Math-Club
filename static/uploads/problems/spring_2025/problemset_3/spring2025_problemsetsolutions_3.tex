\documentclass[11pt, a4paper, oneside]{article}

% ===== Page Layout =====
\usepackage[letterpaper,top=2cm,bottom=2cm,left=3cm,right=3cm,marginparwidth=1.75cm]{geometry}
\usepackage{microtype}  % Improved text justification

% ===== Fonts & Encoding =====
\usepackage[T1]{fontenc}
\usepackage[utf8]{inputenc}
\usepackage[english]{babel}
\usepackage{lmodern}

% ===== Math Packages =====
\usepackage{amsmath, amssymb, amsthm}
\usepackage{stmaryrd}
\usepackage{mathrsfs}
\usepackage{bbm}
\usepackage{tensor}
\usepackage{mathtools}

% ===== Graphics & Diagrams =====
\usepackage{graphicx}
\usepackage{tikz}
\usepackage{tikz-cd}
\usepackage{pgfplots}
\pgfplotsset{compat=1.18}
\usepackage{pst-node}

% ===== Bibliography =====
\usepackage{biblatex}
%\addbibresource{references.bib}  % Uncomment and add your .bib file

% ===== Tables =====
\usepackage{makecell}

% ===== Colors =====
\usepackage{xcolor}
\definecolor{linkcolour}{rgb}{0.5,0,0}  % Dark red color for links

% ===== Hyperlinks =====
\usepackage{hyperref}
\hypersetup{
    colorlinks,
    breaklinks,
    urlcolor=linkcolour, 
    linkcolor=linkcolour,
    citecolor=linkcolour
}

% ===== Custom Commands =====
\newcommand{\problem}[1][]{\section{#1} \hfill \par}
\newcommand{\solution}[1][]{\subsection*{#1}\hfill \par}

% ===== Theorem Environments =====
\newtheorem{theorem}{Theorem}
\theoremstyle{remark}
\newtheorem*{remark}{Remark}
\theoremstyle{lemma}
\newtheorem*{lemma}{Lemma}

% ===== Text Highlighting =====
\usepackage{soul}
\newcommand\ba[1]{\setbox0=\hbox{$#1$}%
\rlap{\raisebox{.45\ht0}{\textcolor{linkcolour}{\rule{\wd0}{1pt}}}}#1} 
\def\bc#1{\textcolor{linkcolour}{BC note: {#1}}}
\def\b#1{\textcolor{linkcolour}{{#1}}}

% ===== Comment Environment =====
\usepackage{comment}
\begin{comment}
Useful LaTeX fonts:
\usepackage{mathptmx}
\usepackage{txfonts}
\usepackage{pxfonts}
\usepackage{mathpazo}
\usepackage{mathpple}
\usepackage{kmath,kerkis}
\usepackage{kurier}
\usepackage{arev}
\usepackage{euler}
\usepackage{eulervm}
\end{comment}

\title{Problem Set Week 3 Solutions}
\author{ETHZ Math Olympiad Club}
\date{10 March 2025}
\begin{document}
\maketitle
\problem[Problem in example page 140 (PUTNAM and BEYOND)]
Let $f : \mathbb{R} \to \mathbb{R}$ be a twice-differentiable function, with positive second derivative. 
Prove that
\[
f(x + f'(x)) \geq f(x),
\]
for any real number $x$.

\solution[Solution:]
If $x$ is such that $f'(x) = 0$, then the relation holds with equality. Else $f'(x)\neq 0$ and then the following set has non empty interior:
$$[\,x + f'(x),\, x\,]\sqcup[\,x,\, x+ f'(x)\,]$$
It is clear that one of these interval is non empty and the second is empty.
The mean value theorem applied on the non empty interval yields the existence of $c\in [\,x + f'(x),\, x\,]\sqcup[\,x,\, x+ f'(x)\,]$ such that:
\[
f'(c) 
= \frac{f\left(x + f'(x)\right) - f(x)}{\left(x + f'(x)\right) - x} 
= \frac{f(x + f'(x)) - f(x)}{f'(x)},
\]
i.e. $f'(c)f'(x)=f(x + f'(x)) - f(x)$.
\\\\
If $f'(x)>0$ then $c\in [\,x,\, x+ f'(x)\,]$ and because the second derivative is positive, $f'$ is increasing; hence $0<f'(x) < f'(c)$. Therefore $f(x + f'(x)) - f(x)>0$.
\\\\
If $f'(x)<0$ then $c\in [\,x+ f'(x),\, x\,]$ and because the second derivative is positive, $f'$ is increasing; hence $f'(c) < f'(x)<0$. Therefore $f(x + f'(x)) - f(x)>0$.
\\\\
In all cases: $$f(x + f'(x)) \geq f(x)$$
\newpage
\problem[Problem A-2 (IMC 2011)]
Does there exist a real $3 \times 3$ matrix $A$ such that 
\[
\operatorname{tr}(A) = 0
\quad\text{and}\quad
A^2 + A^T = I_3,
\]
where $\operatorname{tr}(A)$ denotes the trace of $A$, $A^T$ is the transpose of $A$, and $I_3$ is the $3 \times 3$ identity matrix?

\solution[Solution:]

We claim that no such real $3 \times 3$ matrix $A$ exists. Suppose, for contradiction, that a matrix $A \in \mathbb{R}^{3\times 3}$ exists with $\operatorname{tr}(A) = 0$ and $A^2 + A^T = I_3$.
Taking the transpose of the second equation, we obtain
\[
I_3 = I_3^T = \left(A^2 + A^T\right)^T = \left(A^2\right)^T + A
= \left(A^T\right)^2 + A.
\]
Using the original assumption $A^2 + A^T = I_3$, we substitute:
\[
I_3 = \left(I_3 - A^2\right)^2 + A
= I_3 - A^2 - A^2 + A^4 + A
= A^4 - 2A^2 + A + I_3.
\]
Thus, we obtain the matrix polynomial equation
\[
P(A) = 0_3
\quad\text{with}\quad P(X) := X^4 - 2X^2 + X \in \mathbb{R}[X],
\]
where $0_3$ denotes the $3\times 3$ zero matrix. We now factor the polynomial:
\[
P(X) = X^4 - 2X^2 + X
= X(X-1)\left(X^2 - X - 1\right).
\]
It follows that the minimal polynomial of $A$ must divide $P(X)$, and hence the eigenvalues of $A$ must be among
\[
\left\{0, 1, \tfrac{-1\pm\sqrt{5}}{2}\right\}.
\]
Recalling that the trace of a matrix is the sum of its eigenvalues (counted with multiplicity), we obtain:
\[
0 = \operatorname{tr}(A)
= \sum_{\lambda \in \sigma(A)} \dim_{\mathbb{R}}(\ker(A - \lambda I_3)) \cdot \lambda
\]
where $\sigma(A)$ denotes the spectrum of $A$. Moreover, taking the trace of both sides of $A^2 + A^T = I_3$ gives:
\[
3 = 3 - 0 = \operatorname{tr}(I_3) - \operatorname{tr}(A)
= \operatorname{tr}\left(I_3 - A^T\right)
= \operatorname{tr}\left(A^2\right).
\]
Thus,
\[
\operatorname{tr}(A^2)
= \sum_{\mu \in \sigma(A^2)} \dim_{\mathbb{R}}\left(\ker\left(A^2 - \mu I_3\right)\right) \cdot \mu.
\]
Since $\sigma\left(A^2\right) = \left\{\lambda^2 \mid \lambda \in \sigma(A)\right\}$\footnote{In general, for any $n \times n$ matrix $B$ over $\mathbb{C}$ (or $\mathbb{R}$), write $B$ in its Jordan canonical form
\[
B = V J V^{-1},
\]
where $J$ is a block-diagonal matrix consisting of Jordan blocks corresponding to the eigenvalues $\lambda_1, \dots, \lambda_k$ of $B$, and $V$ is invertible. Applying a polynomial $f(X) \in \mathbb{C}[X]$ to $B$ gives:
\[
f(B)
=
f(V J V^{-1})
=
V f(J) V^{-1}.
\]
By a simple computation, the matrix $f(J)$ is a block-diagonal matrix consisting of Jordan blocks corresponding to the diagonal entries $f(\lambda_1), \dots, f(\lambda_k)$. By uniqueness of the Jordan form (up to block permutation), $f(J)$ is a block-diagonal matrix consisting of Jordan blocks corresponding to the eigenvalues of $f(B)$ showing that the eigenvalues of $f(B)$ are precisely $\{ f(\lambda) \mid \lambda \in \sigma(B)\}$, although the eigenvectors need not coincide.},
we obtain:
\[
3 = \sum_{\lambda \in \sigma(A)} \dim_{\mathbb{R}}\left(\ker\left(A^2 - \lambda^2 I_3\right)\right) \lambda^2.
\]
By direct case-checking of the eigenvalues $\sigma(A) \subset \left\{0, 1, \frac{-1\pm\sqrt{5}}{2}\right\}$, one easily verifies that the two conditions on $\operatorname{tr}(A)$ and $\operatorname{tr}\left(A^2\right)$ cannot be simultaneously satisfied. This yields a contradiction.
\\\\
Hence, no real $3 \times 3$ matrix $A$ satisfies both $\operatorname{tr}(A) = 0$ and $A^2 + A^T = I_3$.

\newpage
\problem[Problem B-2 (IMC 2014)]
Let \( A = (a_{ij})_{i,j=1}^{n} \) be a symmetric \( n \times n \) matrix with real entries, and let \( \lambda_{1}, \lambda_{2}, \ldots, \lambda_{n} \) denote its eigenvalues. Show that
\[
\sum_{1 \leq i < j \leq n} a_{ii} a_{jj} \geq \sum_{1 \leq i < j \leq n} \lambda_{i} \lambda_{j},
\]
and determine all matrices for which equality holds.

\solution[Solution:]
Eigenvalues of a real symmetric matrix are real, hence the inequality is well-defined.
\\
The trace of a matrix equals the sum of its eigenvalues. For matrix \( A \),
\[
\sum_{i=1}^{n} a_{ii} = \sum_{i=1}^{n} \lambda_{i}.
\]
Squaring both sides, we obtain:
\[
\left( \sum_{i=1}^{n} a_{ii} \right)^2 = \left( \sum_{i=1}^{n} \lambda_{i} \right)^2.
\]
Expanding both sides gives:
\[
\sum_{i=1}^{n} a_{ii}^2 + 2 \sum_{1 \leq i < j \leq n} a_{ii} a_{jj} = \sum_{i=1}^{n} \lambda_{i}^2 + 2 \sum_{1 \leq i < j \leq n} \lambda_{i} \lambda_{j}.
\]
It therefore suffices to show the inequality:
\[
\sum_{i=1}^{n} a_{ii}^2 \leq \sum_{i=1}^{n} \lambda_{i}^2.
\]
The matrix \( A^2 \), which equals \( A^T A \) for symmetric \( A \), has eigenvalues \( \lambda_{1}^2, \lambda_{2}^2, \ldots, \lambda_{n}^2 \) for the same reason as the previous problem. The trace of \( A^T A \) is the square of the Frobenius norm of \( A \):
\[
\operatorname{tr}\left(A^T A\right) = \sum_{i,j=1}^{n} a_{ij}^2 = \operatorname{tr}\left(A^2\right) = \sum_{i=1}^{n} \lambda_{i}^2.
\]
Obviously \( \sum_{i=1}^{n} a_{ii}^2 \leq \sum_{i,j=1}^{n} a_{ij}^2 \), the inequality \( \sum_{i=1}^{n} a_{ii}^2 \leq \sum_{i=1}^{n} \lambda_{i}^2 \) follows. One sees then that equality holds if and only if $\sum_{i=1}^{n} a_{ii}^2 = \sum_{i,j=1}^{n} a_{ij}^2$ that is if and only if all off-diagonal entries of \( A \) are zero, i.e., \( A \) is diagonal.
\begin{remark}
The same result holds for Hermitian matrices as for Hermitian matrices, diagonal entries and eigenvalues are also real. 
\end{remark}
\newpage
\problem[Problem 414 (PUTNAM and BEYOND)]
For any real number $\lambda \geq 1$, denote by $f(\lambda)$ the real solution to the equation
\[
x(1 + \ln x) = \lambda.
\]
Prove that
\[
\lim_{\lambda \to +\infty} \frac{f(\lambda)}{\frac{\lambda}{\ln \lambda}} = 1.
\]
\solution[Solution:]
The function \( h : [1, +\infty[ \rightarrow [1, +\infty[\) given by \( h(t) = t(1 + \ln t) \) is strictly increasing, and \( h(1) = 1 \), \(\lim_{t \to +\infty} h(t) = +\infty\). Hence \( h \) is bijective, and its inverse is clearly the function \( f : [1, \infty) \rightarrow [1, \infty) \), \( \lambda \rightarrow f(\lambda) \) satisfying $\lambda=f(\lambda)(1+ln(f(\lambda)))$. Since \( h \) is differentiable with $h'(t)=2+ln(t)$ which never vanishes for $t\in[1; +\infty[$ so is \( f \), and
\[
f'(\lambda) = \frac{1}{h'(f(\lambda))} = \frac{1}{2 + \ln f(\lambda)}.
\]
Also, since \( h \) is strictly increasing and \(\lim_{t \to +\infty} h(t) = +\infty\), \( f(\lambda) \) is strictly increasing, and its limit at + infinity is also + infinity. Using the defining relation for \( f(\lambda) \), we see that for $\lambda \geq 1$:
\[
\frac{f(\lambda)}{\frac{\lambda}{\ln \lambda}} = \ln \lambda \cdot \frac{f(\lambda)}{\lambda} = \frac{\ln \lambda}{1 + \ln f(\lambda)}.
\]
Now we apply L'Hôpital's theorem and obtain
\[
\lim_{\lambda \to +\infty} \frac{f(\lambda)}{\frac{\lambda}{\ln \lambda}} = \lim_{\lambda \to +\infty} \frac{\frac{1}{\lambda}}{\frac{1}{f(\lambda)} \cdot \frac{1}{2 + \ln f(\lambda)}} = \lim_{\lambda \to +\infty} \frac{f(\lambda)}{\lambda} (2 + \ln f(\lambda))\]
\[
= \lim_{\lambda \to +\infty} \frac{2 + \ln f(\lambda)}{1 + \ln f(\lambda)} = 1+\lim_{\lambda \to +\infty}\frac{1}{1+\ln f(\lambda)}=1,
\]
where the last equality follows from the fact $\lim_{\lambda\to +\infty}\ln f(\lambda)=+\infty$. Therefore, the required limit is equal to 1.
\newpage
\problem[Problem A-4 (IMC 2014)]
Let \(n > 6\) be a perfect number, and let \(n = p_{1}^{e_{1}} \cdots p_{k}^{e_{k}}\) be its prime factorisation with \[1 < p_{1} < \ldots < p_{k}.\] Prove that \(e_{1}\) is an even number.\\
A number \(n\) is \textit{perfect} if \(s(n) = 2n\), where \(s(n)=\sum_{\mathbb{N}\ni d|_{\mathbb{Z}}n}d\) is the sum of the divisors of \(n\).
\solution[Solution:]
Suppose that \(e_{1}\) is odd, contrary to the statement. We know that
\[
s(n) = \prod_{i=1}^{k} \left(\sum_{j=0}^{e_i}p_{i}^{j}\right) = 2n = 2p_{1}^{e_{1}} \cdots p_{k}^{e_{k}}.
\]
Since \(e_{1}\) is an odd number, \(p_{1} + 1\) divides the first factor\footnote{\text{one could also see it as follows. Since:} $$\left\{(1+p_{1})\left(\sum_{j=0}^{\frac{e_{1}-1}{2}}(-p_{1})^j\right),(1-p_{1})\left(\sum_{j=0}^{\frac{e_{1}-1}{2}}(-p_{1})^j\right)\right\}=\left\{\left(1-p_{1}^{\frac{e_1+1}{2}}\right),\left(1+p_{1}^{\frac{e_1+1}{2}}\right)\right\}$$\text{where identifying which one is which one depends on the parity of} $\frac{e_1-1}{2}$\text{. We must have that}
\[
\sum_{j=0}^{e_1}p_{1}^{j}=\frac{1-p_{1}^{e_1+1}}{1-p_{1}}=\frac{\left(1-p_{1}^{\frac{e_1+1}{2}}\right)\left(1+p_{1}^{\frac{e_1+1}{2}}\right)}{1-p_{1}}
=\frac{(1+p_{1})\left(\sum_{j=0}^{\frac{e_{1}-1}{2}}(-p_{1})^j\right)(1-p_{1})\left(\sum_{j=0}^{\frac{e_{1}-1}{2}}(-p_{1})^j\right)}{1-p_{1}}
\]
\[
=(1+p_{1})\left(\sum_{j=0}^{\frac{e_{1}-1}{2}}(-p_{1})^j\right)^2.
\]
\text{Comparing with the previous factorization and using its uniqueness yields:}
$$\left(\sum_{j=0}^{\frac{e_1-1}{2}}p_1^{2j}\right)=\left(\sum_{j=0}^{\frac{e_{1}-1}{2}}(-p_{1})^j\right)^2$$}:
$$\sum_{j=0}^{e_1}p_{1}^{j}=\left(\sum_{j=0}^{\frac{e_1-1}{2}}p_1^{2j}\right)+\left(\sum_{j=0}^{\frac{e_1-1}{2}}p_1^{2j+1}\right)=(1+p_1)\left(\sum_{j=0}^{\frac{e_1-1}{2}}p_1^{2j}\right)$$
so \(p_{1} + 1\) divides \(2n\). Due to \(p_{1} + 1 > 2\), at least one of the primes \(p_{1}, \ldots, p_{k}\) divides \(p_{1} + 1\). The primes \(p_{3}, \ldots, p_{k}\) are greater than \(p_{1} + 1\) and \(p_{1}\) cannot divide \(p_{1} + 1\), clearly $p_{1}$ doesn't divide $p_{1}+1$ (else $p_{1}$ would divide $1$) so \(p_{2}\) must divide \(p_{1} + 1\) i.e. $\exists t\in\mathbb{N}^{*}$ with $p_{1}+1=tp_{2}$. Since \(tp_{2}=p_{1} + 1 < 2p_{1}<2p_{2}\), this is possible only if $t=1$ i.e. \(p_{2} = p_{1} + 1\), therefore \(p_{1} = 2\) and \(p_{2} = 3\) (the only two consecutive primes are $2$ and $3$). Hence, \(6 \mid n\).
\\\\
Now \(n, \frac{n}{2}, \frac{n}{3}, \frac{n}{6}\) and \(1\) are distinct divisors of \(n\), so by definition
\[
2n\overset{n\text{ is perfect}}{=}s(n) \geq n + \frac{n}{2} + \frac{n}{3} + \frac{n}{6} + 1 = n\left(1+\frac{1}{2} + \frac{1}{3} + \frac{1}{6}\right)+1= 2n + 1 > 2n,
\]
which is a contradiction.
\begin{remark}
The perfect numbers with a first power odd must therefore be smaller than $6$. One can check that the only perfect number $n\leq 6$ is $6=2\cdot 3$ and this one has an odd first power.
\end{remark}
\end{document}
