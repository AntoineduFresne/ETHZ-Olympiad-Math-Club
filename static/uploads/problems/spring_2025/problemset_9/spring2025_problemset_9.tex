\documentclass[11pt, a4paper, oneside]{article}

% ===== Page Layout =====
\usepackage[letterpaper,top=2cm,bottom=2cm,left=3cm,right=3cm,marginparwidth=1.75cm]{geometry}
\usepackage{microtype}  % Improved text justification

% ===== Fonts & Encoding =====
\usepackage[T1]{fontenc}
\usepackage[utf8]{inputenc}
\usepackage[english]{babel}
\usepackage{lmodern}

% ===== Math Packages =====
\usepackage{amsmath, amssymb, amsthm}
\usepackage{stmaryrd}
\usepackage{mathrsfs}
\usepackage{bbm}
\usepackage{tensor}
\usepackage{mathtools}

% ===== Graphics & Diagrams =====
\usepackage{graphicx}
\usepackage{tikz}
\usepackage{tikz-cd}
\usepackage{pgfplots}
\pgfplotsset{compat=1.18}
\usepackage{pst-node}

% ===== Bibliography =====
\usepackage{biblatex}
%\addbibresource{references.bib}  % Uncomment and add your .bib file

% ===== Tables =====
\usepackage{makecell}

% ===== Colors =====
\usepackage{xcolor}
\definecolor{linkcolour}{rgb}{0.5,0,0}  % Dark red color for links

% ===== Hyperlinks =====
\usepackage{hyperref}
\hypersetup{
    colorlinks,
    breaklinks,
    urlcolor=linkcolour, 
    linkcolor=linkcolour,
    citecolor=linkcolour
}

% ===== Custom Commands =====
\newcommand{\problem}[1][]{\section{#1} \hfill \par}
\newcommand{\solution}[1][]{\subsection*{#1}\hfill \par}

% ===== Theorem Environments =====
\newtheorem{theorem}{Theorem}
\theoremstyle{remark}
\newtheorem*{remark}{Remark}
\theoremstyle{lemma}
\newtheorem*{lemma}{Lemma}

% ===== Text Highlighting =====
\usepackage{soul}
\newcommand\ba[1]{\setbox0=\hbox{$#1$}%
\rlap{\raisebox{.45\ht0}{\textcolor{linkcolour}{\rule{\wd0}{1pt}}}}#1} 
\def\bc#1{\textcolor{linkcolour}{BC note: {#1}}}
\def\b#1{\textcolor{linkcolour}{{#1}}}

% ===== Comment Environment =====
\usepackage{comment}
\begin{comment}
Useful LaTeX fonts:
\usepackage{mathptmx}
\usepackage{txfonts}
\usepackage{pxfonts}
\usepackage{mathpazo}
\usepackage{mathpple}
\usepackage{kmath,kerkis}
\usepackage{kurier}
\usepackage{arev}
\usepackage{euler}
\usepackage{eulervm}
\end{comment}

\title{Problem Set Week 9}
\author{ETHZ Math Olympiad Club}
\date{28 April 2025}
\begin{document}
\maketitle
\problem[Problem (X-ENS 11 Orals)]
Let \(\left(G,\cdot,e_G\right)\) be a finite non-commutative group. Show that the probability that two elements \(x,y\in G\) chosen uniformly at random commute is less than or equal to \(\frac{5}{8}\), and show that this bound is tight, i.e., provide an example where the bound is attained.

\problem[Problem (from the book: Selected Problems in Real Analysis)]
Let \( S \) be a set and \( f: S \rightarrow S \) a bijection. Show that \( f \) can be written as the composition of two involutions, where an involution \( h \) is a function that is its own inverse.

\problem[Problems (Sudakov \& Milojevic)]
\subsection{}
Let $C_{1}, C_{2}, C_{3}$ be disjoint circles in the plane of different radii, and let $T_{ij}$ be the intersection point of the common tangent to $C_{i}$ and $C_{j}$ for all $1 \le i < j \le 3$. Show that the points $T_{12}$, $T_{23}$, and $T_{13}$ lie on a common line.

\subsection{}
Several spherical planets, each of radius $R$, are placed in a greenhouse. On each planet, Mark colors in black the regions that are not visible from any other planet by a single straight-line segment. Prove that the total area of the colored regions, summed over all planets, is exactly $4\pi R^{2}$.

 
\subsection{}
There is a pile of silver coins on a table. John holds two pieces of paper and performs the
following process: at each step, he can add one gold coin to the table and write the current number of silver
coins on one piece of paper, or remove one silver coin from the table and write down the current number
of gold coins on the other piece of paper. This process runs until no more silver coins remain on the table.
Show that at the end of the process, the sums of the numbers on both pieces of paper are equal.

\problem[Problem (Romanian IMO Team Selection Tests 1998)]
Show that the polynomial with integer coefficient $(X^2+X)^{2^n}+1\in\mathbb{Z}[X]$ is irreducible over $\mathbb{Q}[X]$.

\problem[Problem B6 (Putnam 1985)]
Let \( n \geq 1 \) and \( G \leq \mathrm{GL}_n(\mathbb{R}) \) be a finite group consisting of real \( n \times n \) matrices under matrix multiplication. Suppose the sum of the traces of all elements in \( G \) is zero:
\[
\sum_{M \in G} \mathrm{tr}(M) = 0.
\]
Prove that the sum of the elements of \( G \) is the zero matrix:
\[
\sum_{M \in G} M = \mathbf{0}_{n \times n}.
\]

\end{document}