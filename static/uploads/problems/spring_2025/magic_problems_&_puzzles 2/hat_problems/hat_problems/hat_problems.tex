\documentclass[11pt, a4paper, oneside]{article}

% ===== Page Layout =====
\usepackage[letterpaper,top=2cm,bottom=2cm,left=3cm,right=3cm,marginparwidth=1.75cm]{geometry}
\usepackage{microtype}  % Improved text justification

% ===== Fonts & Encoding =====
\usepackage[T1]{fontenc}
\usepackage[utf8]{inputenc}
\usepackage[english]{babel}
\usepackage{lmodern}

% ===== Math Packages =====
\usepackage{amsmath, amssymb, amsthm}
\usepackage{stmaryrd}
\usepackage{mathrsfs}
\usepackage{bbm}
\usepackage{tensor}
\usepackage{mathtools}

% ===== Graphics & Diagrams =====
\usepackage{graphicx}
\usepackage{tikz}
\usepackage{tikz-cd}
\usepackage{pgfplots}
\pgfplotsset{compat=1.18}
\usepackage{pst-node}

% ===== Bibliography =====
\usepackage{biblatex}
\addbibresource{references.bib}  % Uncomment and add your .bib file

% ===== Tables =====
\usepackage{makecell}

% ===== Colors =====
\usepackage{xcolor}
\definecolor{linkcolour}{rgb}{0.5,0,0}  % Dark red color for links

% ===== Hyperlinks =====
\usepackage{hyperref}
\hypersetup{
    colorlinks,
    breaklinks,
    urlcolor=linkcolour, 
    linkcolor=linkcolour,
    citecolor=linkcolour
}

% ===== Custom Commands =====
\newcommand{\problem}[1][]{\section{#1} \hfill \par}
\newcommand{\answer}{\subsection*{Answer:}\hfill \par}

% ===== Theorem Environments =====
\newtheorem{theorem}{Theorem}
\theoremstyle{remark}
\newtheorem*{remark}{Remark}

% ===== Text Highlighting =====
\usepackage{soul}
\newcommand\ba[1]{\setbox0=\hbox{$#1$}%
\rlap{\raisebox{.45\ht0}{\textcolor{linkcolour}{\rule{\wd0}{1pt}}}}#1} 
\def\bc#1{\textcolor{linkcolour}{BC note: {#1}}}
\def\b#1{\textcolor{linkcolour}{{#1}}}

% ===== Comment Environment =====
\usepackage{comment}
\begin{comment}
Useful LaTeX fonts:
\usepackage{mathptmx}
\usepackage{txfonts}
\usepackage{pxfonts}
\usepackage{mathpazo}
\usepackage{mathpple}
\usepackage{kmath,kerkis}
\usepackage{kurier}
\usepackage{arev}
\usepackage{euler}
\usepackage{eulervm}
\end{comment}

\title{Hat Problems}
\author{ETHZ Math Olympiad Club}
\date{5 May 2025}
\begin{document}
\maketitle
\problem[Hat Problems (Unknown)]
\begin{quote}
\textit{A mathematician is a blind man in a dark room looking for a black hat which isn’t there.}
\hfill — Charles Darwin
\end{quote}
In mathematical folklore, there is a long tradition of hat problems. Typically, a group of logicians or mathematicians (or, for some odd reason, gnomes) each wears a hat. They are able to see at least some of the other participants' hats, but not their own. Each player then guesses the colour of \textit{his} own hat, and they all win or lose collectively based on the quality of their guesses. At first glance, it often seems impossible for the participants to win consistently: how should the colour of the other hats help me guess my own? Nonetheless, it is often possible to do surprisingly well. We refer to the book \cite{hardin2013mathematics} for the history and many results on hat problems.
\\\\
The following proposed problems deal with different kinds of hat problems, all considered in countable settings (both the number of mathematicians and the number of colours are countable). The problems ask for the existence of a \textit{strategy} that the mathematicians can use to guess a colour and that should verify some winning property. We let \( \left(1 \leq C \leq \aleph_0\right) \) be the set of colours and \( \left(1 \leq N \leq \aleph_0\right) \) be the set of mathematicians. Of course, one can generalise these games to increase the number of guesses per participant, allow simultaneous or recursive guessing, permit whispering or shouting, or vary how many other hats each mathematician can see, or even to permit uncountable cardinals (but they require one to know the notions of \textit{ordinals} and \textit{cardinals}),
\\\\
The finite problems do not require any advanced mathematics, whereas the countably infinite versions require a bit more machinery, which should be manageable if one has studied naïve set theory up to the Axiom of Choice (\textbf{AC})\footnote{An equivalent form of (\textbf{AC}) is that every collection of non-empty sets has a choice function:
\[
\forall X \left[\left(\forall Y \in X\, \exists z \in Y\right) \to \exists f: X \to \bigcup X \text{ such that } \forall Y \in X,\; f\left(Y\right) \in Y\right].
\]}. For more general results, we advise those familiar with ordinal and cardinal arithmetic to consult the paper \cite{lietz2024infinite}.
\\\\
The players are placed in a finite or infinite line, or in a room sufficiently large. However, we leave the exact logistics of the realisation of these hat games to the imagination of the reader. In the finite case, we assume the players can do anything \textit{finite} (store any finite set, use any finite functions, \ldots). In the infinite case, we assume that each player has a memory capable of storing finitely many sets of cardinality at most \( 2^{\aleph_0} = \left|\mathcal{P}\left(\aleph_0\right)\right| \) and that each of them has perfect eyesight with infinite resolution (or simply that he knows the colour of each mathematician for each permitted index). We also assume, furthermore, that the players have access to the Axiom of Choice (\textbf{AC}) and that the players can use on their own any function that they can store.

\subsection{}
Consider \( 1 \leq N < \aleph_0 \) mathematicians standing in a room and \( 2 \leq C \leq \aleph_0 \) colours. Each mathematician wears a hat whose colour is chosen from the set \( C \). Moreover, each mathematician sees the colours of all the other hats, but not his own. Assume the mathematicians are already ordered, i.e., each is assigned a distinct index \( i \in N \).
\\\\
The game consists of mathematician \( 0 \) shouting \textit{only} a colour (a number) in \( C \) as a guess of his own hat colour (we assume everyone hears it). Then, \textbf{simultaneously}, all other mathematicians \( 1 \leq i < N \) shout a colour in \( C \), as their guess of their own hat colour.
\\\\
Before the game starts and the hats are placed on their heads, the mathematicians can devise a strategy (they are aware of $C$, $N$ and the index they are attributed).

How can they ensure that they are all wrong or all correct?
\\\\
\textbf{Bonus:} Solve the same problem when there are \( N = \aleph_0 \) mathematicians.

\subsection{}
Consider \( 1 \leq N < \aleph_0 \) mathematicians standing in a line and \( 2 \leq C \leq \aleph_0 \) colours. Each mathematician wears a hat whose colour is chosen from the set \( C \). Moreover, each mathematician can see the hat colours of all the mathematicians in front of him, but not his own or those behind him.
\\\\
The game consists of the mathematicians shouting \textbf{recursively} a colour (a number) in \( C \) as a guess of his own hat colour (we assume everyone hears it), starting from the first in line—that is, the one who sees everyone except himself.
\\\\
Before the game begins and the hats are placed on their heads, the mathematicians can devise a strategy (they are aware of \( C \), \( N \), and the index they will occupy in the queue).

How can they ensure that all but at most one mathematician guess his hat colour correctly?
\\\\
\textbf{Bonus:} Solve the same problem when there are \(N = \aleph_0\) mathematicians.

\subsection{}
Consider \(\left.1 \leq N < \aleph_0\right.\) mathematicians standing in a room and \(\left.1 \leq C \leq \aleph_0\right.\) colours. Each mathematician \(\left.0 \leq i < N\right.\) wears a hat whose colour is chosen from the set \(C\). Each mathematician sees the colours of all the other hats but not his own. Assume the mathematicians are already ordered, i.e., each is assigned a distinct index \(i \in N\).
\\\\
The game consists of the mathematicians \textbf{simultaneously} shouting a colour (a number) in \(C\) as a guess of their own hat colour.
\\\\
Before the game starts and the hats are placed on their heads, the mathematicians may devise a strategy (they are aware of \(C\), \(N\), and the index they are assigned).

For which numbers of colours \(\left.1 \leq C \leq \aleph_0\right.\) can they ensure that at least one mathematician guesses his hat colour correctly?
\\\\
\textbf{Bonus:} What can be said, at best, with and without the Axiom of Choice when there are \(\left.1\leq N\leq \aleph_0\right.\) mathematicians and \(\left.1\leq C\leq \aleph_0\right.\) colours?

\subsection{}
In an infinite sequence, countably infinitely many mathematicians stand one behind another. Each mathematician has a natural number on his back, where each number appears exactly once but is assigned arbitrarily to a mathematician (i.e., a permutation \(\pi: \mathbb{N} \to \mathbb{N}\)). Each mathematician can see all the numbers on the backs of those standing in front of him (forming an actual infinite sequence), but not his own number or the numbers of those standing behind him.

\begin{enumerate}
    \item[(a)] The game consists of the mathematicians recursively (starting from the first in line) shouting (we assume that everyone hears, i.e., information propagates recursively to the rest of the queue) \textit{only} a number, which will be interpreted as the guess of the number on his own back. No mathematician is allowed to state a number he sees on any back in front of him, and all are aware of this rule.
    
    How can they ensure that everyone guesses his number correctly?

    \item[(b)] As in (a), but the first \(N \geq 0\) mathematicians are silent (say nothing), meaning the mathematician in the \(N+1\)-th position starts the guessing process.
    
    How can they ensure that everyone except at most \(N\) guesses his number correctly?
\end{enumerate}
\printbibliography

\end{document}