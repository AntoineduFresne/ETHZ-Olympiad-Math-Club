\documentclass[11pt, a4paper, oneside]{article}

% ===== Page Layout =====
\usepackage[letterpaper,top=2cm,bottom=2cm,left=3cm,right=3cm,marginparwidth=1.75cm]{geometry}
\usepackage{microtype}  % Improved text justification

% ===== Fonts & Encoding =====
\usepackage[T1]{fontenc}
\usepackage[utf8]{inputenc}
\usepackage[english]{babel}
\usepackage{lmodern}

% ===== Math Packages =====
\usepackage{amsmath, amssymb, amsthm}
\usepackage{stmaryrd}
\usepackage{mathrsfs}
\usepackage{bbm}
\usepackage{tensor}
\usepackage{mathtools}

% ===== Graphics & Diagrams =====
\usepackage{graphicx}
\usepackage{tikz}
\usepackage{tikz-cd}
\usepackage{pgfplots}
\pgfplotsset{compat=1.18}
\usepackage{pst-node}

% ===== Bibliography =====
\usepackage{biblatex}
%\addbibresource{references.bib}  % Uncomment and add your .bib file

% ===== Tables =====
\usepackage{makecell}

% ===== Colors =====
\usepackage{xcolor}
\definecolor{linkcolour}{rgb}{0.5,0,0}  % Dark red color for links

% ===== Hyperlinks =====
\usepackage{hyperref}
\hypersetup{
    colorlinks,
    breaklinks,
    urlcolor=linkcolour, 
    linkcolor=linkcolour,
    citecolor=linkcolour
}

% ===== Custom Commands =====
\newcommand{\problem}[1][]{\section{#1} \hfill \par}
\newcommand{\solution}[1][]{\subsection*{#1}\hfill \par}

% ===== Theorem Environments =====
\newtheorem{theorem}{Theorem}
\theoremstyle{remark}
\newtheorem*{remark}{Remark}
\theoremstyle{lemma}
\newtheorem*{lemma}{Lemma}

% ===== Text Highlighting =====
\usepackage{soul}
\newcommand\ba[1]{\setbox0=\hbox{$#1$}%
\rlap{\raisebox{.45\ht0}{\textcolor{linkcolour}{\rule{\wd0}{1pt}}}}#1} 
\def\bc#1{\textcolor{linkcolour}{BC note: {#1}}}
\def\b#1{\textcolor{linkcolour}{{#1}}}

% ===== Comment Environment =====
\usepackage{comment}
\begin{comment}
Useful LaTeX fonts:
\usepackage{mathptmx}
\usepackage{txfonts}
\usepackage{pxfonts}
\usepackage{mathpazo}
\usepackage{mathpple}
\usepackage{kmath,kerkis}
\usepackage{kurier}
\usepackage{arev}
\usepackage{euler}
\usepackage{eulervm}
\end{comment}

\title{Problem Set Week 10}
\author{ETHZ Math Olympiad Club}
\date{12 May 2025}
\begin{document}
\maketitle


\problem[Problem B3 (Putnam 2001)]
For each $n\in\mathbb{N}$, let \(\langle n \rangle\) denote the closest integer to $\sqrt{n}$. Find \(\sum_{n=1}^{\infty} \frac{2^{\langle n \rangle} + 2^{-\langle n \rangle}}{2^n}\).

\problem[Problem 3 (Bernoulli Competition 2024)]
Suppose \( x_0 \in \mathbb{R} \) and \( x_{n+1} = \sum_{i=0}^{n} (-1)^i \sin(x_i) \) for \( n \geq 1 \).\\
1) What is the range for the \( x_0 \) such that \( \lim_{n \to \infty} x_n \) exists? What is the value of the limit depending on \( x_0 \) in the range?\\
2) Suppose \( x_0 = 1/4 \). Find \( \lim_{n \to \infty} \dfrac{\log\left(\left| \log(x_n) \right|\right)}{n} \).

\problem[Problem 4 (unknown)]
Let $(S,\cdot)$ be a non-empty magma; that is $S$ be a non-empty set, with an internal binary operation. Suppose that it satisfies the following:\\
-$(S,\cdot)$ forms a semi-group i.e the binary operation $\cdot$ is associative.
-The binary operation $\cdot$ is injective in left and right coordinate $\forall a,b,c\in S$;\\ $$\cdot((a,b))=\cdot((a,c))\Rightarrow b=c\ i.e.\  a\cdot b=a\cdot c\Rightarrow b=c$$ $$\cdot((b,a))=\cdot((c,a))\Rightarrow b=c\ i.e.\ b\cdot a=c\cdot a\Rightarrow b=c$$\\
-$\forall a\in S, a^{\mathbb{Z}_{\geq 1}}:=\{a^n |\ n\in\mathbb{Z}_{\geq 1}\}$ is finite, where $a^1:=a\ and\ for\ n>1,\ a^n=a\cdot a^{n-1}$.
(Note that this definition could be independent where we start to compute since $\cdot$ is associative.)\\\\
Show that $S$ can be turned into a group where the group operation is $\cdot$.

\problem[Problem (unknown)]
There is an odd number ($2 \cdot n + 1 > 0$) of stones with real weights satisfying the following property: if we remove any stone from the $2 \cdot n + 1$, then there is a way to partition the rest of the stones into two sets of size $n$, such that the sum of the weights of the stones in both sets is equal. Show that all stones have the same weight.

\problem[Problem B-6 (IMC 2024)]

Show that any function $f:\mathbb{Q}\longrightarrow\mathbb{Z}$ satisfy the following propertie:
$$\exists a,b,c\in\mathbb{Q}\text{ with } a<b<c \text{ such that } f(a), f(c)\leq f(b)$$


\end{document}
