\documentclass[11pt, a4paper, oneside]{article}

% ===== Page Layout =====
\usepackage[letterpaper,top=2cm,bottom=2cm,left=3cm,right=3cm,marginparwidth=1.75cm]{geometry}
\usepackage{microtype}  % Improved text justification

% ===== Fonts & Encoding =====
\usepackage[T1]{fontenc}
\usepackage[utf8]{inputenc}
\usepackage[english]{babel}
\usepackage{lmodern}

% ===== Math Packages =====
\usepackage{amsmath, amssymb, amsthm}
\usepackage{stmaryrd}
\usepackage{mathrsfs}
\usepackage{bbm}
\usepackage{tensor}
\usepackage{mathtools}

% ===== Graphics & Diagrams =====
\usepackage{graphicx}
\usepackage{tikz}
\usepackage{tikz-cd}
\usepackage{pgfplots}
\pgfplotsset{compat=1.18}
\usepackage{pst-node}

% ===== Bibliography =====
\usepackage{biblatex}
%\addbibresource{references.bib}  % Uncomment and add your .bib file

% ===== Tables =====
\usepackage{makecell}

% ===== Colors =====
\usepackage{xcolor}
\definecolor{linkcolour}{rgb}{0.5,0,0}  % Dark red color for links

% ===== Hyperlinks =====
\usepackage{hyperref}
\hypersetup{
    colorlinks,
    breaklinks,
    urlcolor=linkcolour, 
    linkcolor=linkcolour,
    citecolor=linkcolour
}

% ===== Custom Commands =====
\newcommand{\problem}[1][]{\section{#1} \hfill \par}
\newcommand{\solution}[1][]{\subsection*{#1}\hfill \par}

% ===== Theorem Environments =====
\newtheorem{theorem}{Theorem}
\theoremstyle{remark}
\newtheorem*{remark}{Remark}
\theoremstyle{lemma}
\newtheorem*{lemma}{Lemma}

% ===== Text Highlighting =====
\usepackage{soul}
\newcommand\ba[1]{\setbox0=\hbox{$#1$}%
\rlap{\raisebox{.45\ht0}{\textcolor{linkcolour}{\rule{\wd0}{1pt}}}}#1} 
\def\bc#1{\textcolor{linkcolour}{BC note: {#1}}}
\def\b#1{\textcolor{linkcolour}{{#1}}}

% ===== Comment Environment =====
\usepackage{comment}
\begin{comment}
Useful LaTeX fonts:
\usepackage{mathptmx}
\usepackage{txfonts}
\usepackage{pxfonts}
\usepackage{mathpazo}
\usepackage{mathpple}
\usepackage{kmath,kerkis}
\usepackage{kurier}
\usepackage{arev}
\usepackage{euler}
\usepackage{eulervm}
\end{comment}

\title{Problem Set Week 10 Solutions}
\author{ETHZ Math Olympiad Club}
\date{12 May 2025}
\begin{document}
\maketitle

\problem[Problem B3 (Putnam 2001)]
Let \(\langle n \rangle\) denote the closest integer to $\sqrt{n}$. Find \(\sum_{n=1}^{\infty} \frac{2^{\langle n \rangle} + 2^{-\langle n \rangle}}{2^n}\).

\solution[Solution:]
    \item \textbf{Understanding \(\langle n \rangle\):}  
    For integer \(k\), \(\langle n \rangle = k\) when \((k - 0.5)^2 < n \leq (k + 0.5)^2\). This interval contains \(2k\) integers. Specifically, for \(k \geq 1\), \(\langle n \rangle = k\) for \(n = k^2 - k + 1, \ldots, k^2 + k\).

    \item \textbf{Decomposing the Sum:}  
    Group terms by \(\langle n \rangle = k\):
    \[
    \sum_{n=1}^{\infty} \frac{2^{\langle n \rangle} + 2^{-\langle n \rangle}}{2^n} = \sum_{k=1}^{\infty} \sum_{m=k^2 - k + 1}^{k^2 + k} \frac{2^k + 2^{-k}}{2^m}
    \]

    \item \textbf{Inner Geometric Series:}  
    For fixed \(k\), the inner sum is:
    \[
    \sum_{m=k^2 - k + 1}^{k^2 + k} \frac{1}{2^m} = \frac{1 - 2^{-2k}}{2^{k^2 - k}}
    \]
    Multiply by \(2^k + 2^{-k}\):
    \[
    \frac{(2^k + 2^{-k})(1 - 2^{-2k})}{2^{k^2 - k}} = \frac{2^k - 2^{-k}}{2^{k^2 - k}} - \frac{2^k - 2^{-k}}{2^{k^2 + k}}
    \]

    \item \textbf{Telescoping Sums:}  
    Split into two reindexed sums:
    \[
    \sum_{k=1}^{\infty} \frac{2^k}{2^{k^2 - k}} - \sum_{k=1}^{\infty} \frac{2^{-k}}{2^{k^2 - k}} = \sum_{m=0}^{\infty} 2^{1 - m^2} - \sum_{m=2}^{\infty} 2^{1 - m^2}
    \]
    All intermediate terms cancel, leaving:
    \[
    2^{1 - 0^2} + 2^{1 - 1^2} = 2 + 1 = 3
    \]

\newpage
\problem[Problem 3 (Bernoulli Competition 2024)]
Suppose \( x_0 \in \mathbb{R} \) and \( x_{n+1} = \sum_{i=0}^{n} (-1)^i \sin(x_i) \) for \( n \geq 1 \).\\
1) What is the range for the \( x_0 \) such that \( \lim_{n \to \infty} x_n \) exists? What is the value of the limit depending on \( x_0 \) in the range?\\
2) Suppose \( x_0 = 1/4 \). Find \( \lim_{n \to \infty} \dfrac{\log\left(\left| \log(x_n) \right|\right)}{n} \).
\solution[Solution:]
1) First, note that \( x_1 = \sin(x_0) \), as such \( \left| x_1 \right| \leq 1 \). Second, note that
\[
x_{n+1} = x_n + (-1)^n \sin(x_n), \quad n \geq 2.
\]
Using this recurrent formula, we shall obtain the results.

Notice that \( f(x) = x - \sin(x) \) has the same sign as \( x \), thus the sign of \( x_n \) will be the same as the sign of \( x_1 \). \( x_n \) and \( -x_n \) obey the same recurrent law, so we could assume that \( x_1 \geq 0 \).

Now, given that \( x_{2n-1} \leq 1 \), 
\begin{align*}
x_{2n+1} &= x_{2n} + \sin(x_{2n}) = x_{2n-1} - \sin(x_{2n-1}) + \sin\left(x_{2n-1} - \sin(x_{2n-1})\right) \leq x_{2n-1}
\end{align*}
as such \( x_{2n+1} \) is bounded from above and below, thus there exists a limit. \( x = x - \sin(x) + \sin(x - \sin(x)) \) implies \( \sin(x) = \sin(x - \sin(x)) \), with the limit being between \( 0 \) and \( 1 \). But if \( 1 \geq x > 0 \), \( \sin(x) > \sin(x - \sin(x)) \), thus \( \sin(x) = 0 \), which means \( x = 0 \).

For \( x_{2n}, x_{2n+1} \geq x_{2n} \), thus the second half of the sequence also tends to 0.

Overall, the argument that \( x_n \) tends to 0 is now easy to establish.

2) Notice that 
\[
x_{2n+1} = \dfrac{x_{2n-1}}{3} + o\left(x_{2n-1}^3\right).
\]
One may use the following theorem:

\textbf{Theorem:} Suppose \( \{a_n - a_{n-1}\}_{n \geq 1} \) is a sequence that has a finite limit \( A \). Then for the sequence \( \left\{ \dfrac{a_n}{n} \right\}_{n \geq 1} \), there exists a finite limit \( A \). 

\textbf{Proof:} Assume \( b_n = a_n - a_{n-1}, \, n \geq 1 \), then \( \dfrac{b_1 + b_2 + \dots + b_n}{n} = \dfrac{a_n - a_0}{n} \) is the Cesàro mean of \( b_n \), and since \( b_n \) has a finite limit \( A \), the Cesàro mean of \( b_n \) also has a finite limit \( A \), which implies \( \dfrac{a_n}{n} \) tends to \( A \).

Using this theorem, notice that:
\begin{align*}
\log\left( \left| \log(x_{2n+1}) \right| \right) - \log\left( \left| \log(x_{2n-1}) \right| \right) &= \log\left( \left( \dfrac{ \left| \log(x_{2n+1}) \right| }{ \left| \log(x_{2n-1}) \right| } \right) \right) \\
&= \log\left( \dfrac{3 \log(x_{2n-1}) + O(1)}{\log(x_{2n-1})} \right) = \log(3) \text{ (since } \log(x_{2n-1}) \to -\infty \text{)}
\end{align*}
Thus 
\[
\lim_{n \to \infty} \dfrac{\log\left( \left| \log(x_{2n+1}) \right| \right)}{2n+1} = \dfrac{\log(3)}{2}
\]
Notice that \( x_{2n+1} = x_{2n} + \sin(x_{2n}) \leq 2x_{2n} \).

Finally, using \( x_{2n+1} \geq x_{2n} \geq \dfrac{x_{2n+1}}{2} \), we have 
\[
\left| \log(x_{2n+1}) \right| \leq \left| \log(x_{2n}) \right| \leq \left| \log\left( \dfrac{x_{2n+1}}{2} \right) \right|
\]
for large \( n \), and all these sequences tend to plus infinity, thus
\[
\dfrac{\log\left( \left| \log(x_{2n+1}) \right| \right)}{2n} \leq \dfrac{\log\left( \left| \log(x_{2n}) \right| \right)}{2n} \leq \dfrac{\log\left( \left| \log\left( \dfrac{x_{2n+1}}{2} \right) \right| \right)}{2n}
\]
for large \( n \).

Both left and right sequences tend to \( \dfrac{\log(3)}{2} \), hence the limit applies to \( \dfrac{\log\left( \left| \log(x_{2n}) \right| \right)}{2n} \).\\

Finally, \( y_n = \dfrac{\log\left( \left| \log(x_n) \right| \right)}{n} \), and \( N_1 \) such that \( y_{2n} \) is in \( \varepsilon \)-neighbourhood of \( \dfrac{\log(3)}{2} \) for \( n > N_1 \), \( y_{2n+1} \) is in \( \varepsilon \)-neighbourhood of \( \dfrac{\log(3)}{2} \) for \( n > N_2 \), thus for \( n > \max(2N_1, 2N_2 + 1) \), \( y_n \) is in the \( \varepsilon \)-neighbourhood of \( \dfrac{\log(3)}{2} \), as such the entire sequence tends to \( \dfrac{\log(3)}{2} \).

\newpage
\problem[Problem]
Let $(S,\cdot)$ be a non-empty magma; that is $S$ be a non-empty set, and:
\begin{align*}
\cdot:S\times S &\longrightarrow S\\
(a,b) &\mapsto \cdot((a,b))\overset{not}{=}a\cdot b
\end{align*}
be a function (an internal binary operation).
\\\\
Suppose it satisfies the following:\\
\\
-$(S,\cdot)$ forms a semi-group i.e $\cdot$ is associative; $$\forall a,b,c\in S,\cdot((a,\cdot((b,c)))=\cdot((\cdot(a,b),c))\ i.e.\  a\cdot(b\cdot c)=(a\cdot b)\cdot c$$
-$\cdot$ is injective in left and right coordinate $\forall a,b,c\in S$;\\ $$\cdot((a,b))=\cdot((a,c))\Rightarrow b=c\ i.e.\  a\cdot b=a\cdot c\Rightarrow b=c$$ $$\cdot((b,a))=\cdot((c,a))\Rightarrow b=c\ i.e.\ b\cdot a=c\cdot a\Rightarrow b=c$$\\
-$\forall a\in S, a^{\mathbb{N}_{\geq 1}}:=\{a^n |\ n\in\mathbb{N}_{\geq 1}\}$ is finite, where,
$$a^1:=a\ and\ for\ n>1,\ a^n=a\cdot a^{n-1}$$
Note that this definition is independent where we start to compute since $\cdot$ is associative.\\\\
Show $(S,\cdot)$ is a group.
\solution[Solution:] 
In order to show $(S,\cdot)$ is a group we need to start by showing that $(S,\cdot)$ is a monoïd; that is a semi-group where there exist an identity element for $\cdot$ .
We know $(S,\cdot)$ is a semi-group. We search and identity element.\\\\
Let $a\in S$ which exists since it is non-empty, now we know $a^{\mathbb{N}_{\geq 1}}$ is not infinite which means the following statement is true:
$$
\neg(\forall m',n'\in\mathbb{N}_{\geq 1}, m'< n'\rightarrow a^{m'}\neq a^{n'})
$$
Which means $\exists m,n\in\mathbb{N}_{\geq 1}, m< n\wedge a^m=a^n$. We define:
$$e_S:=a^{n-m>0}\in S$$
We see that since $\cdot$ is an associative binary operation that: $$a=a^1=a^{m-(m-1)}=a^{n-(m-1)}=a\cdot a^{n-m}=a\cdot e_S$$
Therefore again for the same reason we have:  
$$a\cdot e_S^2=a\cdot (e_S\cdot e_S)=(a\cdot e_S)\cdot e_S=a\cdot e_S$$
Using injectivity in left coordinate:
$$e_S^2=e_S$$
This identity will be the key in our context to show $e_S$ is the identity element for $(S,\cdot)$ i.e. $\forall b\in S, e_S\cdot b=b=b\cdot e_S$.\\\\
Let $b\in S$, one has using that $S$ is a semi group: $$b\cdot e_S=b\cdot e_S^2=(b\cdot e_S)\cdot e_S$$
and $$e_S\cdot b=e_S^2\cdot b=e_S\cdot(e_S\cdot b)$$
using injectivity in the right and left coordinate respectively:
$b=b\cdot e_S$ and $b=e_S\cdot b$
as desired.\\
We have therefore shown $(S,\cdot,e_S)$ is a monoïd, to show it is a group we need to show $\forall g\in S,\exists h\in S,g\cdot h=e_S=h\cdot g$ (each element have right and left inverse which are the same).
Let $g\in S$:\\
if $g=e_S$ we have $e_S^2=e_S$ so $e_S$ has an inverse which is itself.
\\
If $g\neq e_S$ then $\exists r,s\in\mathbb{N}_{\geq 1}, r< s\wedge g^r=g^s$:
$$e_S\cdot g=g=g^1=g^{r-(r-1)}=g^{s-(r-1)}=g^{s-r}\cdot g$$
using injectivity in right coordinate:
$g^{s-r}=e_S$ we know $s-r\geq 1$ if $s-r=1$ then $g=e_S$ a contradiction therefore $s-r\geq 2$ so that $s-r-1\geq 1$ and we have:
$$g\cdot g^{s-r-1}=e_S=g^{s-r-1}\cdot g$$
so that $g$ has an inverse which is $g^{s-r-1}$.
Hence $(S,\cdot,e_S)$ is a group.

\newpage
\problem[problem]
(This problem comes from a discussion in a typical hungary restaurant in Buda (Budapest) with Clement Hongler during the MA travel holiday from 10 july to 17 july 2024)
\\\\
There is an odd number ($2 \cdot n + 1 > 0$) of stones with real weights satisfying the following property: if we remove any stone from the $2 \cdot n + 1$, then there is a way to partition the rest of the stones into two sets of size $n$, such that the sum of the weights of the stones in both sets is equal. Show that all stones have the same weight.
\\\\
Reformulated with a first-order formula of two free variables $\mathbf{y}, n$ (obviously in an extension by definition of set theory for the new notation):
$$
\phi(\mathbf{y}, n):\begin{cases}
n \in \mathbb{N} \\
\wedge \\
\mathbf{y}: 2 \cdot n + 1 \to \mathbb{R} \\
\wedge \\
\forall j \in 2 \cdot n + 1 \, \exists \tau_j: 2 \cdot n \overset{\text{bij}}{\rightarrow} 2 \cdot n + 1 \setminus \{j\} \, \sum_{k \in n} \mathbf{y}(\tau_j(k)) = \sum_{t \in 2 \cdot n \setminus n} \mathbf{y}(\tau_j(t))
\end{cases}
$$
We need to show:
$$
\forall \mathbf{y} \forall n \, \phi(\mathbf{y}, n) \rightarrow | \operatorname{ran}(\mathbf{y}) | = 1
$$
\solution[Solution:]

\textit{Solution 1.} Define the weight vector $\mathbf{x}$ and the matrix \( M \in \left\{-1, 0, 1\right\}^{(2n+1) \times (2n+1)} \) whose rows are composed of the vectors \( v_i \), each consisting of \( n \) entries equal to \( 1 \), \( n \) entries equal to \( -1 \), and a single \( 0 \) in position \( i \), in such a way that each row corresponds to a valid stone removal permutation. Clearly, the vector \( \mathbf{1} \) lies in the kernel of \( M \), and by construction, so does the vector \( \mathbf{x} \).
\\\\
We claim that \( \dim\left(\operatorname{Im}(M)\right) = 2n \), so that, by a dimensional argument, \( \ker(M) = \mathbb{R} \mathbf{1} \), hence there exists \( \lambda \in \mathbb{R} \) such that \( \lambda \mathbf{1} = \mathbf{x} \), as desired.
\\\\
From the above observation, we know that \( \dim\left(\operatorname{Im}(M)\right) \leq 2n \). To prove equality, note that \( \dim\left(\operatorname{Im}(M)\right) = \dim\left(\operatorname{Im}\left(M^T\right)\right) \), so it suffices to show that \( \dim\left(\operatorname{Im}\left(M^T\right)\right) = 2n \).
\\\\
Let us show that \( \operatorname{Im}\left(M^T\right) = \left\langle \left\{ v_i \mid i < 2n+1 \right\} \right\rangle \) has dimension \( 2n \). Define \( v_i' := v_i + \mathbf{1} \); this new vector is composed of \( 0 \)s instead of \( -1 \)s, \( 1 \) instead of \( 0 \), and \( 2 \)s instead of \( 1 \).
\\\\
Consider the matrix \( \tilde{M} := M^T + \left( \mathbf{1}, \ldots, \mathbf{1} \right) \). Modulo \( 2 \), this matrix reduces to the identity matrix, i.e., \( \tilde{M} \equiv I \pmod{2} \). Therefore, \( \det\left(\tilde{M}\right) \equiv 1 \pmod{2} \), implying that \( \det\left(\tilde{M}\right) \neq 0 \) and hence \( \tilde{M} \) is invertible. Consequently, \( \mathbb{R}^{2n+1} = \operatorname{Im}\left(\tilde{M}\right) = \left\langle \left\{ v_i' \mid i < 2n+1 \right\} \right\rangle \).
\\\\
Since
\[
\left\{ v_i' \mid i < 2n+1 \right\} \subset \left\langle \left\{ v_i \mid i < 2n+1 \right\} \cup \left\{ \mathbf{1} \right\} \right\rangle,
\]
we have that:
\[
\mathbb{R}^{2n+1} = \mathbb{R} \mathbf{1} + \left\langle \left\{ v_i \mid i < 2n+1 \right\} \right\rangle.
\]
We claim that the sum is direct. Suppose, for contradiction, that \( \mathbf{1} \in \left\langle \left\{ v_i \mid i < 2n+1 \right\} \right\rangle \). Then \( \left\{ v_i' \mid i < 2n+1 \right\} \subset \left\langle \left\{ v_i \mid i < 2n+1 \right\} \right\rangle \), which would imply \( \left\langle \left\{ v_i \mid i < 2n+1 \right\} \right\rangle = \mathbb{R}^{2n+1} \). This implies
\[
2n+1 = \dim\left( \left\langle \left\{ v_i \mid i < 2n+1 \right\} \right\rangle \right) = \dim\left( \operatorname{Im}\left(M^T\right) \right) \leq 2n,
\]
a contradiction. Therefore, \( \mathbf{1} \notin \left\langle \left\{ v_i \mid i < 2n+1 \right\} \right\rangle \). By module theory or vector space theory we have:
\[
\dim\left( \left\langle \left\{ v_i \mid i < 2n+1 \right\}\right\rangle\right) = 2n,
\]
and hence \( \dim\left( \operatorname{Im}(M) \right) = 2n \), as desired.
\\\\
\textit{Solution 2.}
Notice that for any $\mathbf{x}: 2 \cdot n + 1 \to \mathbb{R}$ and $n \in \mathbb{N}$, the property $\phi(\mathbf{x}, n)$ is invariant under multiplication and shift, meaning that if $\alpha \in \mathbb{R}$, then:
$$
\phi(\mathbf{x}, n) \rightarrow \phi(\mathbf{x} + \alpha, n) \wedge \phi(\alpha \cdot \mathbf{x}, n)
$$
$\bullet$ We first start by showing an easier version of the problem:
$$
\forall \mathbf{y} \forall n \left( \operatorname{ran}(\mathbf{y}) \subset \mathbb{Q} \wedge \phi(\mathbf{y}, n) \right) \rightarrow | \operatorname{ran}(\mathbf{y}) | = 1
$$
Let us therefore fix $n \in \mathbb{N}$ and $\mathbf{x}: 2 \cdot n + 1 \to \mathbb{Q}$ with the property $\phi(\mathbf{x}, n)$.

- If $n = 0$ then this is trivial as there is only one stone.

- If $n \geq 1$ then we have two cases:

1. If $\{x_i\}_{i \in 2 \cdot n + 1} = \{0\}$, then we are done.

2. If not, then there is at least one non-zero $x_j \neq 0$ for some $j \in 2 \cdot n + 1$, i.e., say for a subset $\varnothing \neq A \subset 2 \cdot n + 1$ we can write $\{x_i = \frac{p_i}{q_i}\}_{i \in A} \subset \mathbb{Q} \setminus \{0\}$ with $p_i, q_i \in \mathbb{Z}^{*} \times \mathbb{N}^{*}$ and $\gcd(p_i, q_i) = 1$. Then the solution $\phi(\mathbf{x}, n)$ gives us an integer solution with at least one non-zero weight by considering:
$$
\phi(\mathbf{x}':= \text{lcm}(q_i)_{i \in A} \cdot \mathbf{x}, n)
$$
as $0 \neq x'_j \in \{x'_i \mid i \in A\} \subset \operatorname{ran}(\mathbf{x}') \subset \mathbb{N}$.
\\
This leads to a solution with at least one zero-weighted stone by considering:
$$
\phi(\mathbf{x}'' := \mathbf{x}' - \min \operatorname{ran}(\mathbf{x}'), n)
$$
say $x''_t = 0$ for some $t \in 2 \cdot n + 1$. This means $\{0\} \subset \operatorname{ran}(\mathbf{x}'')$.
\\
In fact, we claim $\{0\} = \operatorname{ran}(\mathbf{x}'')$ which will conclude: $\operatorname{ran}(\mathbf{x}) = \{\text{lcm}(q_i)_{i \in A}^{-1} \cdot \min \operatorname{ran}(\mathbf{x}')\}$ and so $\mathbf{x}$ will be a constant function. Therefore, suppose towards a contradiction that
$$
\operatorname{ran}(\mathbf{x}'') \cap \mathbb{Z} \setminus \{0\} \neq \varnothing
$$
This means that the following is well defined:
$$
0 \leq k := \max \left\{ k' \in \mathbb{N} \mid \forall x''_i \in \operatorname{ran}(\mathbf{x}''), 2^{k'} \mid_{\mathbb{Z}} x''_i \right\} \in \mathbb{N}
$$
We have that $\mathbf{z} := 2^{-k} \cdot \mathbf{x}''$ is by construction still an integer number solution with a zero-weighted stone $z_t = x''_t = 0$ and an odd-weighted stone $z_r = 2^{-k} \cdot x''_r \in \operatorname{ran}(\mathbf{z}) \cap 2 \cdot \mathbb{N} + 1$ for a certain (by construction of $k$) $t \neq r \in 2 \cdot n + 1 > 1$ with $2^{k+1} \nmid_{\mathbb{Z}} x''_r$:
$$
0 \in \operatorname{ran}(\mathbf{z}) \wedge \operatorname{ran}(\mathbf{z}) \cap 2 \cdot \mathbb{N} + 1 \neq \varnothing \wedge \phi(\mathbf{z}, n)
$$
From $\phi(\mathbf{z}, n)$ we have choosing $z_t = 0$ and $z_r \in 2 \cdot \mathbb{N} + 1$ respectively that $\exists \tau_t: 2 \cdot n \overset{\text{bij}}{\rightarrow} 2 \cdot n + 1 \setminus \{t\} \exists \tau_r: 2 \cdot n \overset{\text{bij}}{\rightarrow} 2 \cdot n + 1 \setminus \{r\}$ with:
$$
\sum_{l \in n} \mathbf{z}(\tau_t(l)) = \sum_{w \in 2 \cdot n \setminus n} \mathbf{z}(\tau_t(w)) =: E_1 \in \mathbb{N}
$$
and
$$
\sum_{l \in n} \mathbf{z}(\tau_r(l)) = \sum_{w \in 2 \cdot n \setminus n} \mathbf{z}(\tau_r(w)) =: E_2 \in \mathbb{N}
$$
In particular, the total weight of the stones must be at the same time even and odd:
$$
\sum_{i \in 2 \cdot n + 1} z_i = z_t + \sum_{l \in n} \mathbf{z}(\tau_t(l)) + \sum_{w \in 2 \cdot n \setminus n} \mathbf{z}(\tau_t(w)) = 0 + E_1 + E_1 = 2 \cdot E_1 \in 2 \cdot \mathbb{N}
$$
and
$$
\sum_{i \in 2 \cdot n + 1} z_i = z_r + \sum_{l \in n} \mathbf{z}(\tau_r(l)) + \sum_{w \in 2 \cdot n \setminus n} \mathbf{z}(\tau_r(w)) = z_r + E_2 + E_2 = z_r + 2 \cdot E_2 \in 2 \cdot \mathbb{N} + 1
$$
a contradiction with:
$$
\left(2 \cdot \mathbb{N}\right) \cap \left(2 \cdot \mathbb{N} + 1\right) = \varnothing
$$
Therefore $\{0\}=\operatorname{ran}(\textbf{x}'')$ i.e. $\textbf{x}$ is constant. Since $n$ and $\textbf{x}$ were arbitrary we indeed have $$\forall \textbf{x} \forall n \left( \operatorname{ran}(\textbf{x}) \subset \mathbb{Q} \wedge \phi(\textbf{x}, n) \right) \rightarrow |\operatorname{ran}(\textbf{x})| = 1$$
\\\\
$\bullet$ Having solved for rational weights, we now use this for the general case, i.e., $\operatorname{ran}(\textbf{x}) \subset \mathbb{R}$. The idea is to apply $dim_{\mathbb{Q}}(\langle \operatorname{ran}(\textbf{x}) \rangle_{\mathbb{Q}})$, the result for the rational case, to certain rational weights. We will express each $x_i$ as a unique $\mathbb{Q}$-linear combination of elements in a certain basis of $\operatorname{ran}(\textbf{x})$. Conceptually, this will form a matrix with rational coefficients, where the $i$-th row (indexed by the basis) corresponds to those unique weighted rational coefficients in the expression of $x_i$. By using the fact that $\textbf{x}$ satisfies the property, we will deduce that each column of this matrix gives us a set of rational weighted coefficients which also satisfies the property. Therefore, by applying the result for the rational case on each column, we deduce that each of them consists of only one rational number. From this, we then easily see that each $x_i$ must be equal. 
\\\\
We formalize this more properly: Let us therefore fix $n \in \mathbb{N}$ and $\mathbf{x}: 2 \cdot n + 1 \to \mathbb{R}$ with the property $\phi(\mathbf{x}, n)$.

- If $n = 0$ then this is trivial as there is only one stone.

- If $n \geq 1$ then we have two cases:

1. If $\{x_i\}_{i \in 2 \cdot n + 1} = \{0\}$, then we are done.

2. If not, then there is at least one non-zero $x_j \neq 0$ for some $j \in 2 \cdot n + 1$, in this case we choose a subset $\varnothing\subsetneq S \subset \{x_i\}_{i \in 2n + 1}$ with the property:
\[
\langle \{x_i\}_{i \in 2n + 1} \rangle_{\mathbb{Q}} = \bigoplus_{x_i \in S} \mathbb{Q} \cdot x_i
\]
(i.e., a $\mathbb{Q}$-basis of $\langle \{x_i\}_{i \in 2n + 1} \rangle_{\mathbb{Q}}$ within $\{x_i\}_{i \in 2n + 1}$. To do it more formally, we simply note that the set of free $\mathbb{Q}$-families in $\operatorname{ran}(\textbf{x})$ is non-empty as it contains $\{x_j\}$, and so it suffices to take one free family with maximal cardinality; such a family must be a basis).
\\\\
Now, for each $i \in 2n + 1$, we know by construction of the $\mathbb{Q}$-basis that there exists $\sigma_i: S \rightarrow \mathbb{Q}$ with
\[
x_i = \sum_{x_j \in S} \sigma_i(x_j) \cdot x_j
\]
i.e., $x_i$ is a $\mathbb{Q}$-linear combination of the elements of the $\mathbb{Q}$-basis $S$ (these represent the rows of the matrix). To get the columns of the matrix, we define for each $x_j \in S$:
\[
\Delta_{x_{j}}(i) := \sigma_i(x_j)
\]
Let us show that for each $x_h \in S$, $\phi(\Delta_{x_{h}}, n)$. Take $l \in 2n + 1$. Since $\phi(\textbf{x}, n)$, we have that $\exists \tau_l : 2n \overset{\text{bij}}{\rightarrow} 2n + 1 \setminus \{l\}$ with:
\[
\sum_{k \in n} x_{\tau_l(k)} = \sum_{t \in 2n \setminus n} x_{\tau_l(t)}
\]
which becomes, by definition, substitution, and switching the finite sums:
\begin{align*}
\sum_{x_j \in S} \sum_{k \in n} \Delta_{x_{j}}(\tau_{l}(k)) \cdot x_j &= \sum_{x_j \in S} \sum_{k \in n} \sigma_{\tau_{l}(k)}(x_j) \cdot x_j \\
&= \sum_{k \in n} \sum_{x_j \in S} \sigma_{\tau_{l}(k)}(x_j) \cdot x_j \\
&= \sum_{t \in 2n \setminus n} \sum_{x_j \in S} \sigma_{\tau_{l}(t)}(x_j) \cdot x_j \\
&= \sum_{x_j \in S} \sum_{t \in 2n \setminus n} \sigma_{\tau_{l}(t)}(x_j) \cdot x_j \\
&= \sum_{x_j \in S} \sum_{k \in 2n \setminus n} \Delta_{x_{j}}(\tau_{l}(k)) \cdot x_j
\end{align*}
Since $S$ is free, we must have the uniqueness of the coefficients, for each $x_j \in S$:
\[
\sum_{k \in n} \Delta_{x_{j}}(\tau_{l}(k)) = \sum_{k \in 2n \setminus n} \Delta_{x_{j}}(\tau_{l}(k))
\]
and since $l \in 2n + 1$ was arbitrary, we conclude $\forall x_j \in S$ that $\phi(\Delta_{x_{j}}, n)$. Since for each $x_j \in S$ we have $\operatorname{ran}(\Delta_{x_{j}}) \subset \mathbb{Q}$, we may apply $|S| = dim_{\mathbb{Q}}(\langle \operatorname{ran}(\textbf{x}) \rangle_{\mathbb{Q}})$ times the result for rational weights to obtain that each $\Delta_{x_{j}}$ is constant and therefore we have for each $d, e \in 2n + 1$:
\[
\sigma_d(x_j) = \Delta_{x_{j}}(d) = \Delta_{x_{j}}(e) = \sigma_e(x_j)
\]
Therefore, for each $d, e \in 2n + 1$:
\[
x_d - x_e = \sum_{x_j \in S} \sigma_{d}(x_j) \cdot x_j - \sum_{x_j \in S} \sigma_{e}(x_j) \cdot x_j = \sum_{x_j \in S} (\sigma_{d}(x_j) - \sigma_{e}(x_j)) \cdot x_j = 0
\]
So for each $d, e \in 2n + 1$ we have $x_e = x_d$. This concludes that $|\operatorname{ran}(\textbf{x})| = 1$.
\newpage
\problem[Problem B-6 (IMC 2024)]

Show that any function $f:\mathbb{Q}\longrightarrow\mathbb{Z}$ satisfy the following propertie:
$$\exists a,b,c\in\mathbb{Q}\text{ with } a<b<c \text{ and } f(b)\geq f(a)\wedge f(b)\geq f(c)$$
i.e. show that given any $f$:
$$\phi((\mathbb{Q},<),f):\Big(f:\mathbb{Q}\longrightarrow\mathbb{Z}\Big)\rightarrow\exists (a,b,c)\left((a,b,c)\in\mathbb{Q}^{3}\wedge a<b<c\wedge f(b)\geq f(c)\wedge f(b)\geq f(a)\right)$$
\solution[Solution:]
look at the solutions I did (complete it woth the paper)


\end{document}
