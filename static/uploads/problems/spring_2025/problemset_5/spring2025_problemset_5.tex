\documentclass[11pt, a4paper, oneside]{article}

% ===== Page Layout =====
\usepackage[letterpaper,top=2cm,bottom=2cm,left=3cm,right=3cm,marginparwidth=1.75cm]{geometry}
\usepackage{microtype}  % Improved text justification

% ===== Fonts & Encoding =====
\usepackage[T1]{fontenc}
\usepackage[utf8]{inputenc}
\usepackage[english]{babel}
\usepackage{lmodern}

% ===== Math Packages =====
\usepackage{amsmath, amssymb, amsthm}
\usepackage{stmaryrd}
\usepackage{mathrsfs}
\usepackage{bbm}
\usepackage{tensor}
\usepackage{mathtools}

% ===== Graphics & Diagrams =====
\usepackage{graphicx}
\usepackage{tikz}
\usepackage{tikz-cd}
\usepackage{pgfplots}
\pgfplotsset{compat=1.18}
\usepackage{pst-node}

% ===== Bibliography =====
\usepackage{biblatex}
%\addbibresource{references.bib}  % Uncomment and add your .bib file

% ===== Tables =====
\usepackage{makecell}

% ===== Colors =====
\usepackage{xcolor}
\definecolor{linkcolour}{rgb}{0.5,0,0}  % Dark red color for links

% ===== Hyperlinks =====
\usepackage{hyperref}
\hypersetup{
    colorlinks,
    breaklinks,
    urlcolor=linkcolour, 
    linkcolor=linkcolour,
    citecolor=linkcolour
}

% ===== Custom Commands =====
\newcommand{\problem}[1][]{\section{#1} \hfill \par}
\newcommand{\answer}{\subsection*{Answer:}\hfill \par}

% ===== Theorem Environments =====
\newtheorem{theorem}{Theorem}
\theoremstyle{remark}
\newtheorem*{remark}{Remark}

% ===== Text Highlighting =====
\usepackage{soul}
\newcommand\ba[1]{\setbox0=\hbox{$#1$}%
\rlap{\raisebox{.45\ht0}{\textcolor{linkcolour}{\rule{\wd0}{1pt}}}}#1} 
\def\bc#1{\textcolor{linkcolour}{BC note: {#1}}}
\def\b#1{\textcolor{linkcolour}{{#1}}}

% ===== Comment Environment =====
\usepackage{comment}
\begin{comment}
Useful LaTeX fonts:
\usepackage{mathptmx}
\usepackage{txfonts}
\usepackage{pxfonts}
\usepackage{mathpazo}
\usepackage{mathpple}
\usepackage{kmath,kerkis}
\usepackage{kurier}
\usepackage{arev}
\usepackage{euler}
\usepackage{eulervm}
\end{comment}

\title{Problem Set Week 5}
\author{ETHZ Math Olympiad Club}
\date{24 March 2025}
\begin{document}
\maketitle
\problem[Problem (unknown)]
We consider a game where two indistinguishable envelopes are presented to a player:
\begin{itemize}
    \item One envelope contains an amount $\alpha \in \mathbb{R}_{>0}$.
    \item The other envelope contains $2\alpha$.
\end{itemize}

The game proceeds as follows:
\begin{enumerate}
    \item The player randomly selects one envelope (with equal probability).
    \item The player observes the content $x$ of the selected envelope (without knowing $\alpha$).
    \item The player must decide whether to:
    \begin{itemize}
        \item Keep the current envelope, or
        \item Switch to the other envelope (this decision is irrevocable).
    \end{itemize}
\end{enumerate}
Although the game is played once, the player's objective is still to maximize their \textit{expected gain}. Assuming access to \textit{randomness}, how can they do better than always keeping the first envelope?

\problem[Problem A-3 (IMC 2018)]
Determine all rational numbers $a$ for which the matrix
\[
A = \begin{bmatrix}
    a & a & 1 & 0 \\
    -a & -a & 0 & 1 \\
    -1 & 0 & a & a \\
    0 & -1 & -a & -a
\end{bmatrix}
\]
is the square of a matrix with all rational entries.

\problem[Problem A-4 (IMC 2005)]
Find all polynomials
\[
P(X) = a_n X^n + a_{n-1} X^{n-1} + \dots + a_1 X + a_0 \quad (a_n \neq 0)
\]
satisfying the following two conditions:

\begin{enumerate}
    \item \( (a_0, a_1, \dots, a_n) \) is a permutation of the numbers \( (0, 1, \dots, n) \), and
    \item all roots of \( P(X) \) are rational numbers.
\end{enumerate}

\problem[Problem A-6 (IMC 2005)]
Let $m,n\in\mathbb{Z}$. Given a group \( G \), denote by \( G(m) \) the subgroup generated by the \( m \)-th powers of elements of \( G \):
\[
G(m):=\left\langle\left\{g^m \mid g\in G\right\}\right\rangle\leq G.
\]
If \( G(m) \) and \( G(n) \) are commutative, prove that \( G(\gcd(m, n)) \) is also commutative. Here, \( \gcd(m, n) \) denotes the greatest common divisor of \( m \) and \( n \).

\end{document}