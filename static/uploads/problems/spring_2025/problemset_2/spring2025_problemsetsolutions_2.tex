\documentclass[11pt, a4paper, oneside]{article}

% ===== Page Layout =====
\usepackage[letterpaper,top=2cm,bottom=2cm,left=3cm,right=3cm,marginparwidth=1.75cm]{geometry}
\usepackage{microtype}  % Improved text justification

% ===== Fonts & Encoding =====
\usepackage[T1]{fontenc}
\usepackage[utf8]{inputenc}
\usepackage[english]{babel}
\usepackage{lmodern}

% ===== Math Packages =====
\usepackage{amsmath, amssymb, amsthm}
\usepackage{stmaryrd}
\usepackage{mathrsfs}
\usepackage{bbm}
\usepackage{tensor}
\usepackage{mathtools}

% ===== Graphics & Diagrams =====
\usepackage{graphicx}
\usepackage{tikz}
\usepackage{tikz-cd}
\usepackage{pgfplots}
\pgfplotsset{compat=1.18}
\usepackage{pst-node}

% ===== Bibliography =====
\usepackage{biblatex}
%\addbibresource{references.bib}  % Uncomment and add your .bib file

% ===== Tables =====
\usepackage{makecell}

% ===== Colors =====
\usepackage{xcolor}
\definecolor{linkcolour}{rgb}{0.5,0,0}  % Dark red color for links

% ===== Hyperlinks =====
\usepackage{hyperref}
\hypersetup{
    colorlinks,
    breaklinks,
    urlcolor=linkcolour, 
    linkcolor=linkcolour,
    citecolor=linkcolour
}

% ===== Custom Commands =====
\newcommand{\problem}[1][]{\section{#1} \hfill \par}
\newcommand{\solution}[1][]{\subsection*{#1}\hfill \par}

% ===== Theorem Environments =====
\newtheorem{theorem}{Theorem}
\theoremstyle{remark}
\newtheorem*{remark}{Remark}
\theoremstyle{lemma}
\newtheorem*{lemma}{Lemma}

% ===== Text Highlighting =====
\usepackage{soul}
\newcommand\ba[1]{\setbox0=\hbox{$#1$}%
\rlap{\raisebox{.45\ht0}{\textcolor{linkcolour}{\rule{\wd0}{1pt}}}}#1} 
\def\bc#1{\textcolor{linkcolour}{BC note: {#1}}}
\def\b#1{\textcolor{linkcolour}{{#1}}}

% ===== Comment Environment =====
\usepackage{comment}
\begin{comment}
Useful LaTeX fonts:
\usepackage{mathptmx}
\usepackage{txfonts}
\usepackage{pxfonts}
\usepackage{mathpazo}
\usepackage{mathpple}
\usepackage{kmath,kerkis}
\usepackage{kurier}
\usepackage{arev}
\usepackage{euler}
\usepackage{eulervm}
\end{comment}

\title{Problem Set Wee 2 Solutions}
\author{ETHZ Math Olympiad Club}
\date{3 March 2025}
\begin{document}
\maketitle
\problem[Problem 1 (Pan African 2018)]
Find all functions $f : \mathbb Z \to \mathbb Z$ such that $$(f(x + y))^2 = f(x^2) + f(y^2)$$ for all $x, y \in \mathbb Z$.
\solution[Solution:]
Plug in $y=-x$, and let $c=f(0)$. We have $f(x^2) = \frac{c^2}{2}$. Plugging in $y=0$, we have $f(x)^2 = \frac{c^2}{2} + c$. Plugging in $x=0$ in this gives $c^2=\frac{c^2}{2}+c$ $c=0$ or $c=2$. Now if $c=0$, $f \equiv 0$ is the only solution. Else we have $f(x) = \pm 2$ for all $x$. This means $f(x^2) + f(y^2) = 4$, always, so for all $x$ being perfect squares, $f(x)=2$. So the solution in this case becomes $f(x) = 2$ for all perfect square $x$ and $\pm 2$ for all other $x$.
It is easy to check that the solutions mentioned above work. Thus:
\[
\left\{ f : \mathbb{Z} \to \mathbb{Z} \,\middle|\, \forall x,y \in \mathbb{Z} \ (f(x + y))^2 = f(x^2) + f(y^2) \right\}
\]
\[
=
\left\{ \underline{0} \right\} \cup \left\{ 2\mathbbm{1}_{A} - 2\mathbbm{1}_{\mathbb{Z} \setminus A} \,\middle|\, \exists A \subset \mathbb{Z}, \ \square_{\mathbb{N}} \subset A \right\}
\]
where \(\square_{\mathbb{N}} := \{ n^2 \mid n \in \mathbb{N} \}\).

\newpage
\problem[Problem B-2 (IMC 2012)]
Define the sequence \( (a_n)_{n \geq 0} \) inductively by \( a_0 = 1 \), \( a_1 = \frac{1}{2} \), and  
\[
a_{n+1} = \frac{n a_n^2}{1 + (n+1)a_n} \quad \text{for } n \geq 1.
\]
Show that the series  
\[
\sum_{k=0}^{\infty} \frac{a_{k+1}}{a_k}
\]
converges and determine its value.

\solution[Solution:]  
By induction, we establish that \( a_i > 0 \) for all \( i \geq 0 \), ensuring that the partial sums are well defined and form an increasing sequence. Furthermore, we observe that  
\[
k a_k = \frac{(1 + (k+1)a_k) a_{k+1}}{a_k} = \frac{a_{k+1}}{a_k} + (k+1)a_{k+1} \quad \text{for all } k \geq 1.
\]
Summing from \( k = 0 \) to \( n \), we obtain  
\[
\sum_{k=0}^{n} \frac{a_{k+1}}{a_k} = \frac{a_1}{a_0} + \sum_{k=1}^{n} (k a_k - (k+1)a_{k+1}) = \frac{1}{2} + 1 \cdot a_1 - (n+1)a_{n+1} = 1 - (n+1)a_{n+1}
\]
for all \( n \geq 1 \). A quick verification confirms that this equality also holds for \( n = 0 \).  
\\\\
Since \( (n+1)a_{n+1} > 0 \), we deduce that for each \( n \geq 0 \),  
\[
\sum_{k=0}^{n} \frac{a_{k+1}}{a_k} = 1 - (n+1)a_{n+1} < 1.
\]
This implies that the partial sums are bounded, and thus, the series \( \sum_{k=0}^{\infty} \frac{a_{k+1}}{a_k} \) is convergent (converging to the supremum of the partial sums). Consequently, the sequence \( \frac{a_{k+1}}{a_k} \) must tend to zero. In particular, there exists an index \( N \in \mathbb{N} \) such that \( \frac{a_{n+1}}{a_n} < \frac{1}{2} \) for all \( n > N \).  
\\\\
For all \( n > N \), we then have  
\[
a_n = \prod_{i=N}^{n-1} \frac{a_{i+1}}{a_i} < \frac{1}{a_N 2^{n-1-N}} = \frac{a_N^{-1} 2^N + 1}{2^n}.
\]
In particular, for all \( n > N \),  
\[
n a_n < \left( a_N^{-1} 2^N + 1 \right) \left( \frac{n}{2^n} \right).
\]
Since \( \frac{n}{2^n} \to 0 \) as \( n \to +\infty \), it follows that \( n a_n \to 0 \) as $ n \to +\infty$. Therefore,  
\[
\sum_{k=0}^{\infty} \frac{a_{k+1}}{a_k} = \lim_{n \to \infty} \sum_{k=0}^{n} \frac{a_{k+1}}{a_k} = \lim_{n \to \infty} \left( 1 - (n+1)a_{n+1} \right) = 1.
\]

\newpage
\problem[Problem 5 (Pan African 2018)]
Let $a$, $b$, $c$ and $d$ be non-zero pairwise different real numbers such that
$$
  \frac{a}{b} + \frac{b}{c} + \frac{c}{d} + \frac{d}{a} = 4 \text{ and } ac = bd.
$$
Show that
$$
  \frac{a}{c} + \frac{b}{d} + \frac{c}{a} + \frac{d}{b} \leq -12
$$and that $-12$ is the maximum.
\solution[Solution:]
First write $d=\frac{ac}{b}$ so we have $\frac{a}{b}+\frac{b}{a}+\frac{b}{c}+\frac{c}{b}=4$ and we want to show that $\frac{a}{c}+\frac{c}{a}+\frac{b^2}{ac}+\frac{ac}{b^2}$ or equivalently $(\frac{a}{b}+\frac{b}{a})(\frac{c}{b}+\frac{b}{c}) \leqslant -12$.
\\\\
Set $\frac{a}{b}+\frac{b}{a}=s$ and $\frac{c}{b}+\frac{b}{c}=t$. We have $s+t=4$ and we want $st \leqslant -12$.
\\\\
We have $t=4-s$ and so we need $s(4-s) \leqslant -12 \Rightarrow s^2-4s-12 \geqslant 0$. If $s \leqslant -2$, this holds (indeed, set $s=-2+k$ and we have $s^2-4s-12=(k-2)^2-4(k-2)-12=k^2-8k-16=(k-4)^2 \geqslant 0$).
\\\\
So assume that $s \geqslant -2$. Similarly, assume $t \geqslant -2$. We have then $\frac{a}{b}+\frac{b}{a} \geqslant -2 \Rightarrow \frac{(a+b)^2}{ab}>0$, so $ab>0$ and $bc>0$, so $a,b,c,$ are of the same sign, so $4=\frac{a}{b}+\frac{b}{a}+\frac{b}{c}+\frac{c}{b} \geqslant 2+2=4$, so $a=b=c$, contradiction, since $a \neq b \neq c \neq a$.
\\\\
So, at least one of $s,t$ is $\leqslant -2$ and the stated inequality holds.
\\\\
The equality happens e.g. when $(a,b,c,d)=(2\sqrt{2}-3,3-2\sqrt{2},1,-1)$ (and this shows that $-12$ is the maximum).
\newpage
\problem[Problem 3 (Silk Road 2019)]
Find all pairs \( (a, n) \) of positive natural numbers such that \( \varphi \left( a^n + n \right) = 2^n \).

(\( \varphi(n) \) is the Euler function, that is, the number of integers from \(1\) up to \( n \), relatively prime to \( n \).)

\solution[Solution:]
Using the estimation \(\varphi(b) > \sqrt{b}\) (classic) for all \(b > 6\), we first check the (finitely many) cases when a solution \((a,n)\) satisfies \(a^n + n \le 6\) to find that only:
\[
(a,n) \in \left\{ (2,1), (3,1), (5,1) \right\}
\]
works.
\\\\
Now for a solution \((a,n)\) with \(a^n + n > 6\) we get using the lower bound estimation:
\[
\varphi \left( a^n + n \right) \overset{\text{hyp}}{=} 2^n > \sqrt{a^n + n}.
\]
\(\bullet\) If \(a \ge 4\) the inequality cannot be true as then:
\[
2^n > \sqrt{a^n + n} \geq \sqrt{4^n + n} \geq \sqrt{2^{2n}} = 2^n,
\]
so we have to check the cases when \(a = 1, 2, 3\).
\\\\
\(\bullet\) If \(a = 1\) then using the trivial upper bound estimation \(\varphi(b) \leq b - 1\) we get \(n \geq \varphi(n + 1) = 2^n\), which never holds for \(n \geq 1\).
\\
\(\bullet\) If \(a = 2\) or \(a = 3\) this is more subtle. To have \(3^n + n > 6\) we need for \(a = 3\) that \(n \geq 2\) and for \(a = 2\) that \(n \geq 3\). We see that \((3,3)\) works: indeed \(\varphi(30) = 30 \left(1 - \frac{1}{2}\right) \left(1 - \frac{1}{3}\right) \left(1 - \frac{1}{5}\right) = 8 = 2^3\). Since \((3,2)\) and \((2,3)\) don't satisfy the property (easy to check) and \((3,3)\) does, we can assume \(n \geq 4\). Now we claim that there is no integer \(n \geq 4\) satisfying for \(a \in \{2,3\}\):
\[
\varphi \left( a^n + n \right) = 2^n.
\]
We proceed as follows:
\\\\
If an integer \( m \geq 2 \) satisfies
\[
\varphi(m) = 2^k,
\]
then every odd prime \( p \) dividing \( m \) must satisfy
\[
p - 1 = 2^j.
\]
Indeed, writing \(m = 2^k \cdot p_1^{e_1} \cdots p_r^{e_r}\) for some distinct odd primes \(p_i\), \(k \geq 0\) and \(e_1, \ldots, e_r \geq 1\), since \(\varphi(m) = 2^{k-1} \cdot p_1^{e_1 - 1} \cdots p_r^{e_r - 1} (p_1 - 1) \cdots (p_r - 1)\), the only way for \(\varphi(m)\) to be a power of \(2\) is that:
\begin{itemize}
    \item For each \(i \in \{1, \ldots, r\}\), we have \(e_i = 1\).
    \item Every odd prime \(p_i\) satisfies \(p_i - 1 = 2^{a_i}\) for some \(a_i \geq 0\). In other words, every odd prime factor of \( m \) is a Fermat prime\footnote{A Fermat prime is a prime number of the form \(2^j + 1\). One can easily see (mental exercise) that \(j\) is necessarily \textbf{again} a power of \(2\). Indeed, if \(j\) is not a power of \(2\), then there is an odd prime \(p \mid j\) and \(2^j + 1 = \left(2^{\frac{j}{p}}\right)^p - (-1)^p = \left(2^{\frac{j}{p}} + 1\right) \left( \sum_{s=0}^{p-1} (-1)^{p-1-s} \left(2^{\frac{j}{p}}\right)^s \right)\), which is a non-trivial factorization of \(2^j + 1\) since \(1 < 2^{\frac{j}{p}} + 1 < 2^j + 1\), contradicting the fact that \(2^j + 1\) is a prime number. \textit{Nota bene}: the only known Fermat primes are \(3 = 2^{(2^0)} + 1\), \(5 = 2^{(2^1)} + 1\), \(17 = 2^{(2^{2})} + 1\), \(257 = 2^{(2^{3})} + 1\), and \(65537 = 2^{(2^{4})} + 1\), and we don't know if there are infinitely many of them.}.
\end{itemize}
Hence \(2^n = \varphi(m) = 2^{k-1} \prod_{i=1}^r 2^{a_i}\), and so the following equality holds:
\[
k - 1 + \sum_{i=1}^r a_i = n.
\]
If we let \(n \geq 4\), \(a \in \{2,3\}\), and
\[
m = a^n + n,
\]
with \(\varphi(m) = 2^n\), then with the same notation as before:
\[
a^n + n = 2^k \cdot p_1 \cdots p_r = 2^k \left(2^{a_1} + 1\right) \cdots \left(2^{a_r} + 1\right),
\]
for \(k \geq 0\) and some distinct odd primes \(p_i = 2^{a_i} + 1\) with \(a_i \geq 1\) (as \(p_i\) is odd) (the \(a_i\) are powers of \(2\), but this is not needed here). Moreover, the equality \(k - 1 + \sum_{i=1}^r a_i = n\) holds. Without loss of generality, order the exponents:
\[
1 \leq a_1 < \cdots < a_r.
\]
Now let us attack each case ($a=2$ or $a=3$ with $n\geq 4$) separately:
\\\\
-If \(a = 2\), then if \(6 < 2^n + n\) is a prime number, this means that \(k = 0\) and \(r = 1\), and \(-1 + a_1 = n\), so that \(2^n + n = 2^{a_1} + 1 = 2^{n+1} + 1\), and thus \(n = 2^n + 1 > 2^n > n\), which is a contradiction. Thus, \(2^n + n\) satisfying the property cannot be a prime number. Hence, it is composite. We have the following classical upper bound for the Euler totient for a composite natural number \(m\):
\[
\varphi(m) = m \prod_{p \mid m} \left(1 - \frac{1}{p}\right) \leq m \left(1 - \frac{1}{\min \left\{ p \in \mathbb{P} \mid p \mid m \right\}}\right) = m - \frac{m}{\min \left\{ p \in \mathbb{P} \mid p \mid m \right\}}.
\]
And since \(m\) is composite (we use it here), \(\min \left\{ p \in \mathbb{P} \mid p \mid m \right\} \leq \sqrt{m}\), so that:
\[
\varphi(m) \leq m - \sqrt{m}.
\]
Using this upper bound estimation, we get:
\[
2^n + n - \sqrt{2^n + n} \geq \varphi \left(2^n + n\right) = 2^n.
\]
Noticing that the function \(f\) defined on \(\mathbb{R}_+\) by \(f(x) = x - \sqrt{2^x + x}\) is always negative (because for all $x\geq 0$ we have $x(x-1)<x^2<2^x$ since $\ln(x)<\frac{x}{2}$), we conclude that the above inequality cannot hold for \(n \geq 1\). This shows that there is no integer \(n \geq 4\) such that \(\varphi(2^n + n) = 2^n\).
\\\\
-If \(a = 3\), then by induction, one has the following fact:
\[
t \geq 1 \implies 2^t + 1 \leq 3^t \text{ with equality only at } t = 1,
\]
\[
t \geq 3 \implies 2^t < 2^t + 1 < 3^{t-1}.
\]
With this in mind, we obtain easily:
\\\\
If there exists \(a_i\) with \(a_i \geq 3\) (in particular, this occurs if \(r \geq 3\), as then for \(r \geq i > 2\) we have \(a_i \geq 3\)), then \(2^{a_i} + 1 < 3^{a_i - 1}\), which yields:
\[
3^n < 3^n + n = 2^k \left(2^{a_1} + 1\right) \cdots \left(2^{a_r} + 1\right) < 2^k 3^{a_i - 1} \prod_{\substack{t=1 \\ t \neq i}}^r 3^{a_t} = 2^k 3^{n - k} \leq 3^n,
\]
which is a contradiction. Thus, in particular, \(1 \leq r \leq 2\), and all \(a_i\) belong to \(\{1,2\}\).
\\\\
If \(k \geq 3\), then:
\[
3^n < 3^n + n = 2^k \left(2^{a_1} + 1\right) \cdots \left(2^{a_r} + 1\right) \leq 2^k 3^{a_1} \cdots 3^{a_r} \leq 2^k 3^{n + 1 - k} \leq 3^{k - 1} 3^{n + 1 - k} = 3^n,
\]
which is again a contradiction, so \(0 \leq k \leq 2\).
\\\\
Summarizing, we have that a solution $3^n+n$ must be in the following set:
\[
\left\{3^j + j \mid j \in \mathbb{N}, j \geq 4\right\} \cap \left\{2^k \left(2^a + 1\right)^{\epsilon_a} \left(2^b + 1\right)^{\epsilon_b} \mid k \in \{0,1,2\}, a,b \in \{1,2\}, \epsilon_a, \epsilon_b \in \{0,1\}, a \neq b \right\}
\]
\[
= \left\{3^j + j \mid j \in \mathbb{N}, j \geq 4\right\} \cap \left\{2^k 3^{\epsilon} 5^{\epsilon'} \mid k \in \{0,1,2\}, \epsilon, \epsilon' \in \{0,1\} \right\}
\]
\[
= \{85\} \cap \left\{1, 3, 5, 9, 15, 25, 2, 6, 10, 18, 30, 50, 4, 12, 20, 36, 60, 100\right\} = \varnothing.
\]
This shows that there is no integer \(n \geq 4\) such that \(\varphi(3^n + n) = 2^n\).
\\\\
To summarize, the only pairs that work are \(\left\{(2,1), (3,1), (3,3), (5,1)\right\}\), and this concludes.
\end{document}
