\documentclass[11pt, a4paper, oneside]{article}

% ===== Page Layout =====
\usepackage[letterpaper,top=2cm,bottom=2cm,left=3cm,right=3cm,marginparwidth=1.75cm]{geometry}
\usepackage{microtype}  % Improved text justification

% ===== Fonts & Encoding =====
\usepackage[T1]{fontenc}
\usepackage[utf8]{inputenc}
\usepackage[english]{babel}
\usepackage{lmodern}

% ===== Math Packages =====
\usepackage{amsmath, amssymb, amsthm}
\usepackage{stmaryrd}
\usepackage{mathrsfs}
\usepackage{bbm}
\usepackage{tensor}
\usepackage{mathtools}

% ===== Graphics & Diagrams =====
\usepackage{graphicx}
\usepackage{tikz}
\usepackage{tikz-cd}
\usepackage{pgfplots}
\pgfplotsset{compat=1.18}
\usepackage{pst-node}

% ===== Bibliography =====
\usepackage{biblatex}
%\addbibresource{references.bib}  % Uncomment and add your .bib file

% ===== Tables =====
\usepackage{makecell}

% ===== Colors =====
\usepackage{xcolor}
\definecolor{linkcolour}{rgb}{0.5,0,0}  % Dark red color for links

% ===== Hyperlinks =====
\usepackage{hyperref}
\hypersetup{
    colorlinks,
    breaklinks,
    urlcolor=linkcolour, 
    linkcolor=linkcolour,
    citecolor=linkcolour
}

% ===== Custom Commands =====
\newcommand{\problem}[1][]{\section{#1} \hfill \par}
\newcommand{\answer}{\subsection*{Answer:}\hfill \par}


% ===== Theorem Environments =====
\newtheorem{theorem}{Theorem}
\theoremstyle{remark}
\newtheorem*{remark}{Remark}

% ===== Text Highlighting =====
\usepackage{soul}
\newcommand\ba[1]{\setbox0=\hbox{$#1$}%
\rlap{\raisebox{.45\ht0}{\textcolor{linkcolour}{\rule{\wd0}{1pt}}}}#1} 
\def\bc#1{\textcolor{linkcolour}{BC note: {#1}}}
\def\b#1{\textcolor{linkcolour}{{#1}}}

% ===== Comment Environment =====
\usepackage{comment}
\begin{comment}
Useful LaTeX fonts:
\usepackage{mathptmx}
\usepackage{txfonts}
\usepackage{pxfonts}
\usepackage{mathpazo}
\usepackage{mathpple}
\usepackage{kmath,kerkis}
\usepackage{kurier}
\usepackage{arev}
\usepackage{euler}
\usepackage{eulervm}
\end{comment}

\title{Problem Set Week 1}
\author{ETHZ Math Olympiad Club}
\date{24 February 2025}
\begin{document}
\maketitle

\problem[Problem 1 (First IMO 1956)]
Prove that the fraction \(\frac{21n+4}{14n+3}\) is irreducible for every natural number \(n\).

\problem[Problem 5 (Russian Math Olympiad 1961)]
a) Given a 4-tuple of positive numbers \((a,b,c,d)\), it is transformed into a new one according to the rule:
    \[
    a' = ab, \quad b' = bc, \quad c' = cd, \quad d' = da.
    \]
The second 4-tuple is transformed into the third according to the same rule, and so on.
\\\\
Prove that if at least one initial number does not equal 1, then you can never obtain the initial 4-tuple again.
\\\\
b) Given a \(2^k\)-tuple of numbers (where \(k\) is a nonnegative integer), each equal either to 1 or to -1, the tuple is transformed according to the same rule as in part (a): each number is multiplied by the next one, and the last is multiplied by the first.
\\\\
Prove that you will always eventually obtain a tuple consisting entirely of 1s.
    
\problem[Problem A-3 (Putnam 1989)]
Prove that if the complex number \( z \) satisfies the equation

\[
11z^{10} + 10i z^9 + 10i z -11 = 0,
\]

then \( |z| = 1 \). (Here, \( i \) is the imaginary unit satisfying \( i^2 = -1 \).)

\problem[Problem 4 (IMC 1994)]
Let \( \alpha \in \mathbb{R} \setminus \{0\} \) and suppose that \( F \) and \( G \) are linear maps (operators) from \( \mathbb{R}^n \) into \( \mathbb{R}^n \) satisfying 
\[
F \circ G - G \circ F = \alpha F.
\]

a) Show that for all \( k \in \mathbb{N} \) one has 
\[
F^k \circ G - G \circ F^k = \alpha k F^k.
\]

b) Show that there exists \( k \geq 1 \) such that \( F^k = 0 \).


\problem[Problem A-6 (Putnam 1989)]

Let 
\[
\alpha(X) = 1 + a_1 X + a_2 X^2 + \cdots \in \mathbb{F}_2[[X]]
\]
be a formal power series with coefficients in the field of two elements. Define the coefficients \( a_n \) as follows:

\[
a_n =
\begin{cases} 
1, & \text{if every block of zeros in the binary expansion of } n \text{ has an even number of zeros} \\
0, & \text{otherwise}.
\end{cases}
\]

For example, \( a_{36} = 1 \) because \( 36 = 100100_2 \), and \( a_{20} = 0 \) because \( 20 = 10100_2 \). 

Prove that the power series satisfies the equation:

\[
\alpha(X)^3 + X \alpha(X) + 1 = 0.
\]

\end{document}
