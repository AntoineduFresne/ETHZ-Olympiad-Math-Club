\documentclass[11pt, a4paper, oneside]{article}

% ===== Page Layout =====
\usepackage[letterpaper,top=2cm,bottom=2cm,left=3cm,right=3cm,marginparwidth=1.75cm]{geometry}
\usepackage{microtype}  % Improved text justification

% ===== Fonts & Encoding =====
\usepackage[T1]{fontenc}
\usepackage[utf8]{inputenc}
\usepackage[english]{babel}
\usepackage{lmodern}

% ===== Math Packages =====
\usepackage{amsmath, amssymb, amsthm}
\usepackage{stmaryrd}
\usepackage{mathrsfs}
\usepackage{bbm}
\usepackage{tensor}
\usepackage{mathtools}

% ===== Graphics & Diagrams =====
\usepackage{graphicx}
\usepackage{tikz}
\usepackage{tikz-cd}
\usepackage{pgfplots}
\pgfplotsset{compat=1.18}
\usepackage{pst-node}

% ===== Bibliography =====
\usepackage{biblatex}
%\addbibresource{references.bib}  % Uncomment and add your .bib file

% ===== Tables =====
\usepackage{makecell}

% ===== Colors =====
\usepackage{xcolor}
\definecolor{linkcolour}{rgb}{0.5,0,0}  % Dark red color for links

% ===== Hyperlinks =====
\usepackage{hyperref}
\hypersetup{
    colorlinks,
    breaklinks,
    urlcolor=linkcolour, 
    linkcolor=linkcolour,
    citecolor=linkcolour
}

% ===== Custom Commands =====
\newcommand{\problem}[1][]{\section{#1} \hfill \par}
\newcommand{\solution}[1][]{\subsection*{#1}\hfill \par}

% ===== Theorem Environments =====
\newtheorem{theorem}{Theorem}
\theoremstyle{remark}
\newtheorem*{remark}{Remark}
\theoremstyle{lemma}
\newtheorem*{lemma}{Lemma}

% ===== Text Highlighting =====
\usepackage{soul}
\newcommand\ba[1]{\setbox0=\hbox{$#1$}%
\rlap{\raisebox{.45\ht0}{\textcolor{linkcolour}{\rule{\wd0}{1pt}}}}#1} 
\def\bc#1{\textcolor{linkcolour}{BC note: {#1}}}
\def\b#1{\textcolor{linkcolour}{{#1}}}

% ===== Comment Environment =====
\usepackage{comment}
\begin{comment}
Useful LaTeX fonts:
\usepackage{mathptmx}
\usepackage{txfonts}
\usepackage{pxfonts}
\usepackage{mathpazo}
\usepackage{mathpple}
\usepackage{kmath,kerkis}
\usepackage{kurier}
\usepackage{arev}
\usepackage{euler}
\usepackage{eulervm}
\end{comment}

\title{Problem Set Week 1 Solutions}
\author{ETHZ Math Olympiad Club}
\date{24 February 2025}
\begin{document}
\maketitle

\problem[Problem 1 (First IMO 1956)]
Prove that the fraction \(\frac{21n+4}{14n+3}\) is irreducible for every natural number \(n\).

\solution[Solution:]
By Bezout's Lemma, $3 \cdot (14n+3) - 2 \cdot (21n + 4) = 1$, so the GCD of the numerator and denominator is $1$ and the fraction is irreducible.
\newpage
\problem[Problem 5 (Russian Math Olympiad 1961)]
a) Given a 4-tuple of positive numbers \((a,b,c,d)\), it is transformed into a new one according to the rule:
    \[
    a' = ab, \quad b' = bc, \quad c' = cd, \quad d' = da.
    \]
    The second 4-tuple is transformed into the third according to the same rule, and so on.
\\\\
Prove that if at least one initial number does not equal 1, then you can never obtain the initial 4-tuple again.
\\\\
b) Given a \(2^k\)-tuple of numbers (where \(k\) is a nonnegative integer), each equal either to 1 or to -1, the tuple is transformed according to the same rule as in part (a): each number is multiplied by the next one, and the last is multiplied by the first.
\\\\
Prove that you will always eventually obtain a tuple consisting entirely of 1s.

\solution[Solution:]
\begin{enumerate}
    \item[(a)] Let \( Q_0 = (a, b, c, d) \) be the original 4-tuple and let \( Q_n \) be the 4-tuple after \( n \) transformations. We consider the product of all elements:
    \[
    abcd.
    \]
    If \( abcd > 1 \), then at each step, the product of the transformed values is strictly increasing, making a return to the initial 4-tuple impossible. Similarly, if \( abcd < 1 \), the product is strictly decreasing, again preventing a return. Therefore, the only possibility for a return is if \( abcd = 1 \).

    Next, define \( M(Q) \) as the largest number in the 4-tuple \( Q \). If at least one number in \( Q_1 \) is not \( 1 \), then necessarily \( M(Q_1) > 1 \). Moreover, in \( Q_3 \), each element is the square of some element from \( Q_1 \) (use $abcd=1$), implying:
    \[
    M(Q_3) = M(Q_1)^2.
    \]
    Since squaring a number greater than 1 results in an even larger number, the sequence \( M(Q_1), M(Q_3), M(Q_5), \dots \) grows indefinitely. This precludes any possibility of returning to the original 4-tuple, as a return would imply cycling, contradicting the unbounded growth of \( M(Q_n) \).

    \item[(b)] After \( r \) transformations, the first element of the \( n \)-tuple can be written as:
    \[
    a_1 \prod_{i=1}^{r} a_{i+1}^{\binom{r}{i}},
    \]
    where \( \binom{r}{i} \) is the binomial coefficient. By induction, after \( n = 2^k \) transformations, this simplifies to:
    \[
    a_1^2 \prod_{i=1}^{n-1} a_{i+1}^{\binom{n}{i}}.
    \]
    It suffices to prove that \( \binom{n}{i} \) is even for all \( 0 < i < n \) when \( n \) is a power of \( 2 \).  

    This follows from the fact that in Pascal's triangle, the binomial coefficient \( \binom{n}{i} \) is given by:
    \[
    \binom{n}{i} = \frac{n!}{i!(n-i)!}.
    \]
    Since \( n = 2^k \), the highest power of \( 2 \) dividing \( n \) is greater than the highest power of \( 2 \) dividing \( i \) or \( n-i \), ensuring that \( \binom{n}{i} \) is always even for \( 0 < i < n \). Thus, all elements in the sequence eventually become \( 1 \), completing the proof.
\end{enumerate}
    
\problem[Problem A-3 (Putnam 1989)]
Prove that if the complex number \( z \) satisfies the equation

\[
11z^{10} + 10i z^9 + 10i z -11 = 0,
\]

then \( |z| = 1 \). (Here, \( i \) is the imaginary unit satisfying \( i^2 = -1 \).)

\solution[Solution:]

We begin by expressing $z^9$ in terms of $z$:
\begin{equation*}
    z^9 = \frac{11 - 10iz}{11z + 10i}.
\end{equation*}
Taking the modulus on both sides, we examine the numerator and denominator separately:
\begin{align*}
    |11 - 10iz|^2 &= (11 - 10iz)(11 + 10iz) = 121 + 100|z|^2 + 220 \Im z, \\
    |11z + 10i|^2 &= (11z + 10i)(11\bar{z} - 10i) = 100 + 121|z|^2 + 220 \Im z.
\end{align*}
If we define $|z^9| = N/D$ where
\begin{equation*}
    N = 121 + 100|z|^2 + 220 \Im z, \quad D = 100 + 121|z|^2 + 220 \Im z,
\end{equation*}
then their difference is given by
\begin{equation*}
    N - D = 21(1 - |z|^2).
\end{equation*}

Now, consider the cases:
\\\\
- If $|z| > 1$, then $1 - |z|^2 < 0$, so $N < D$ and hence $|z^9| < 1$.
\\\\
- If $|z| < 1$, then $1 - |z|^2 > 0$, so $N > D$ and hence $|z^9| > 1$.

Both cases is a contradiction. Thus, we conclude that $|z| = 1$ for all roots $z$.

\newpage
\problem[Problem 4 (IMC 1994)]
Let \( \alpha \in \mathbb{R} \setminus \{0\} \) and suppose that \( F \) and \( G \) are linear maps (operators) from \( \mathbb{R}^n \) into \( \mathbb{R}^n \) satisfying 
\[
F \circ G - G \circ F = \alpha F.
\]

a) Show that for all \( k \in \mathbb{N} \) one has 
\[
F^k \circ G - G \circ F^k = \alpha k F^k.
\]

b) Show that there exists \( k \geq 1 \) such that \( F^k = 0 \).
\solution[Solution:]
\begin{enumerate}
    \item[(a)] We prove the result by induction on $k$. The base case $k = 1$ is given by the assumption. Assume that the formula holds for some $k \geq 1$, i.e.,
    \[
    F^k \circ G - G \circ F^k = \alpha k F^k.
    \]
    We compute:
    \begin{align*}
        F^{k+1} \circ G - G \circ F^{k+1} 
        &= F^k \circ(F \circ G) - G \circ F^k \circ F \\
        &= F^k \circ(G \circ F + \alpha F) - G \circ F^k\circ F \\
        &= F^k\circ G \circ F + \alpha F^{k+1} - G \circ F^{k}\circ F \\
        &= (F^k \circ G - G\circ F^k) \circ F + \alpha F^{k+1} \\
        &= (\alpha k F^k) \circ F + \alpha F^{k+1} \\
        &= \alpha k F^{k+1} + \alpha F^{k+1} \\
        &= \alpha (k+1) F^{k+1}.
    \end{align*}
    This completes the induction step, proving the result.

    \item[(b)] Consider the linear operator $L$ defined on the space of $n \times n$ matrices by
    \[
    L(F) = F \circ G - G \circ F.
    \]
    By the given assumption, we have $L(F) = \alpha F$. Also using part (a), we obtain
    \[
    L(F^k) = \alpha k F^k.
    \]
    The operator $L$ acts on an $n^2$-dimensional space and can have at most $n^2$ distinct eigenvalues. However, if $F^k \neq 0$ for all $k$, then $L$ has infinitely many eigenvalues $\alpha k$, which contradicts the fact that the space is finite-dimensional. Hence, there must exist some $k \geq 1$ such that $F^k = 0$.
\end{enumerate}
\newpage
\problem[Problem A-6 (Putnam 1989)]

Let 
\[
\alpha(X) = 1 + a_1 X + a_2 X^2 + \cdots \in \mathbb{F}_2[[X]]
\]
be a formal power series with coefficients in the field of two elements. Define the coefficients \( a_n \) as follows:

\[
a_n =
\begin{cases} 
1, & \text{if every block of zeros in the binary expansion of } n \text{ has an even number of zeros} \\
0, & \text{otherwise}.
\end{cases}
\]

For example, \( a_{36} = 1 \) because \( 36 = 100100_2 \), and \( a_{20} = 0 \) because \( 20 = 10100_2 \). 

Prove that the power series satisfies the equation:

\[
\alpha(X)^3 + X \alpha(X) + 1 = 0.
\]

\solution[Solution:]
We begin by analyzing the properties of the coefficients \( a_n \). Since the field has characteristic 2, we have \( \alpha(X)^2 = 1 + \sum a_n X^{2n} \). This is because squaring a power series in characteristic 2 eliminates the cross terms, and \( a_n^2 = a_n \) (as \( a_n \in \{0, 1\} \)).
\\\\
Next, consider \( \alpha(X)^3 \). Multiplying the given equation \( \alpha(X)^3 + X \alpha(X) + 1 = 0 \) by \( \alpha(X) \), we obtain:
\[
\alpha(X)^4 + X \alpha(X)^2 + \alpha(X) = 0.
\]
We now analyze \( \alpha(X)^4 + X \alpha(X)^2 + \alpha(X) \) by examining the coefficients of \( X^{4n} \), \( X^{4n+2} \), and \( X^{2n+1} \).
\begin{itemize}
    \item \textbf{Coefficient of \( X^{4n} \):} The coefficient is \( a_n + a_{4n} \). Since the binary expansion of \( 4n \) is the binary expansion of \( n \) with two zeros appended at the end, \( a_{4n} = a_n \). Thus, \( a_n + a_{4n} = 0 \) in characteristic 2. For \( n = 0 \), the coefficient is \( 1 + 1 = 0 \).
    
    \item \textbf{Coefficient of \( X^{4n+2} \):} The coefficient is \( a_{4n+2} \). The binary expansion of \( 4n+2 \) ends with \( \dots 10 \), so \( a_{4n+2} = 0 \).

    \item \textbf{Coefficient of \( X^{2n+1} \):} The coefficient is \( a_{2n+1} + a_n \). The binary expansion of \( 2n+1 \) is the binary expansion of \( n \) with an extra 1 appended at the end. Thus, \( a_{2n+1} = a_n \), and \( a_{2n+1} + a_n = 0 \).
\end{itemize}
Since all coefficients of \( \alpha(X)^4 + X \alpha(X)^2 + \alpha(X) \) are zero, we have:
\[
\alpha(X)^4 + X \alpha(X)^2 + \alpha(X) = 0.
\]
Since \( \alpha(X) \neq 0 \) and $\mathbb{F}_{2}[[X]]$ is a domain as $\mathbb{F}_{2}$ is a domain, we must obtain:
\[
\alpha(X)^3 + X \alpha(X) + 1 = 0,
\]
as required.

\end{document}