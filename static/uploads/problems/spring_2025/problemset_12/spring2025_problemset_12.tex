\documentclass[11pt, a4paper, oneside]{article}

% ===== Page Layout =====
\usepackage[letterpaper,top=2cm,bottom=2cm,left=3cm,right=3cm,marginparwidth=1.75cm]{geometry}
\usepackage{microtype}  % Improved text justification

% ===== Fonts & Encoding =====
\usepackage[T1]{fontenc}
\usepackage[utf8]{inputenc}
\usepackage[english]{babel}
\usepackage{lmodern}

% ===== Math Packages =====
\usepackage{amsmath, amssymb, amsthm}
\usepackage{stmaryrd}
\usepackage{mathrsfs}
\usepackage{bbm}
\usepackage{tensor}
\usepackage{mathtools}

% ===== Graphics & Diagrams =====
\usepackage{graphicx}
\usepackage{tikz}
\usepackage{tikz-cd}
\usepackage{pgfplots}
\pgfplotsset{compat=1.18}
\usepackage{pst-node}

% ===== Bibliography =====
\usepackage{biblatex}
%\addbibresource{references.bib}  % Uncomment and add your .bib file

% ===== Tables =====
\usepackage{makecell}

% ===== Colors =====
\usepackage{xcolor}
\definecolor{linkcolour}{rgb}{0.5,0,0}  % Dark red color for links

% ===== Hyperlinks =====
\usepackage{hyperref}
\hypersetup{
    colorlinks,
    breaklinks,
    urlcolor=linkcolour, 
    linkcolor=linkcolour,
    citecolor=linkcolour
}

% ===== Custom Commands =====
\newcommand{\problem}[1][]{\section{#1} \hfill \par}
\newcommand{\solution}[1][]{\subsection*{#1}\hfill \par}

% ===== Theorem Environments =====
\newtheorem{theorem}{Theorem}
\theoremstyle{remark}
\newtheorem*{remark}{Remark}
\theoremstyle{lemma}
\newtheorem*{lemma}{Lemma}

% ===== Text Highlighting =====
\usepackage{soul}
\newcommand\ba[1]{\setbox0=\hbox{$#1$}%
\rlap{\raisebox{.45\ht0}{\textcolor{linkcolour}{\rule{\wd0}{1pt}}}}#1} 
\def\bc#1{\textcolor{linkcolour}{BC note: {#1}}}
\def\b#1{\textcolor{linkcolour}{{#1}}}

% ===== Comment Environment =====
\usepackage{comment}
\begin{comment}
Useful LaTeX fonts:
\usepackage{mathptmx}
\usepackage{txfonts}
\usepackage{pxfonts}
\usepackage{mathpazo}
\usepackage{mathpple}
\usepackage{kmath,kerkis}
\usepackage{kurier}
\usepackage{arev}
\usepackage{euler}
\usepackage{eulervm}
\end{comment}

\title{Problem Set Week 12}
\author{ETHZ Math Olympiad Club}
\date{26 May 2025}
\begin{document}
\maketitle
\problem[Problem (unknown)]
Let $n\in\mathbb{Z}_{>0}$, prove that the decimal part of
\[
\frac{5 + \sqrt{26}}{n}
\]
begins with either $n$ zeros or $n$ nines for all positive integers $n$.



\problem[Problem 1 (Bernoulli Competition 2024)]
Let \( A, B \in \mathrm{Mat}_{n \times n}(\mathbb{C}) \). Suppose that every matrix \( C \in \mathrm{Mat}_{n \times n}(\mathbb{C}) \) can be written in the form \( C = AD - DB \) for some \( D \in \mathrm{Mat}_{n \times n}(\mathbb{C}) \). Prove that there exists a polynomial \( R \in \mathbb{C}[x] \) such that
\[
R(A) = A^3 - A^2 - A + I_n \quad \text{and} \quad R(B) = (2 - i)B^4 + (2 + i)B^3 - B.
\]




\problem[Problem (unknown)]
There are $1\leq N\in\mathb{N}$ prisoners are imprisoned in solitary confinement. Each cell is soundproof and windowless. There is a central living room with a single light bulb, initially turned off. The prisoners cannot see the bulb from their cells.
\\\\
Each day, the warden selects one prisoner uniformly at random to visit the central room. While in the room, the prisoner may toggle the bulb (on $\leftrightarrow$ off) if they so choose. The prisoner may also choose to declare that \textit{all $N$ prisoners have visited the central room at least once}. If this claim is false (i.e., at least one prisoner has never been in the room), all $N$ prisoners are executed. If the claim is true, they are all freed and inducted into MENSA.
\\\\
Before this process begins, the prisoners are allowed to meet once in a courtyard to devise a strategy. What strategy should they adopt to ensure their eventual release, with certainty?
\\\\
The strategy must guarantee eventual success, not merely a high probability.
\\\\
The average time until release depends on the strategy.
\\\\
The best-known strategies achieve an average release time of approximately 3500 days.



\problem[Problem (Hongler)]
A princess has $1 \leq N$ suitors. Each suitor possesses a distinct ``score''. Among them is Prince Charming, who has the maximal score. The princess encounters the suitors in a random order. For each suitor, she must decide either to marry him or to reject him (and, if she rejects him, she cannot go back). She can compare the score of the current suitor with the scores of those she has already rejected, but otherwise she cannot recognise Prince Charming. She wishes to maximise her chance of marrying Prince Charming (if she reaches the end of the list of $N$ suitors, she must marry the last one).\\[0.5em]
\textbf{(a)} How can she find a strategy that guarantees a chance $\geq \frac{1}{4}$ of marrying Prince Charming?\\
\textbf{(b)} What is the optimal strategy? \emph{Hint:} it gives a chance $>36\%$. \emph{Hint:} the strategy is not complicated at all.\\


\problem[problem (Bernoulli Competition 2025)]
Let $\alpha\in\mathbb{R}$ such that $\forall n\in\mathbb{Z}_{>0}\, n^\alpha:=\exp(\alpha\cdot\ln(n))\in\mathbb{Z}_{>0}$. Show that $\alpha\in\mathbb{N}$.

\end{document}
